%===================================================================
% Latex file: 11.dec.2016,Exam style:
%===================================================================
%===================================================================
% By default LaTeX uses large margins.  This doesn't work well on exams; problems
% end up in the "middle" of the page, reducing the amount of space for students
% to work on them.
% For an exam, single spacing is most appropriate
%===================================================================

\documentclass[11pt]{exam}
\RequirePackage{amssymb, amsfonts, amsmath, latexsym, verbatim, xspace, setspace}
\usepackage[margin=1in]{geometry}
\singlespacing

%===================================================================

\begin{document}
\title{\textbf{Estimation : Interesting questions and Back of the envelope calculations}}
\maketitle

%===================================================================
\subsection{Avogadro Number: A Mole.}
%===================================================================

\begin{questions}

%===================================================================
\question
\label{Q1:Al Foil}

How long one has to roll a Aluminium foil to get 1 mole of Aluminium atoms?\\
\textbf{Answer }: \\
\textbf{Step 1}. Required information\\ 

			     Atomic mass of Aluminium = 27. This means 1 mole of    Aluminium(Al) atoms weigh 27 grams.\\
			     Density of Al = 2.7 grams/$cm^{3}$\\
			     Dimensions of household Al foil:\\ Length = 25 meter(2500cm), Width = 30cm, and Thickness = 14 microns(0.0014cm).\\ 
		                                    
\fbox{\begin{minipage}{28em} 
                 
\textbf{Step 2}. Volume of Al foil for above dimensions.\\

  Volume = Width $\times$ Length $\times$ Thickness\\
         = (30 $\times$ 0.014 $\times$ 2500) $cm^{3}$\\
         = 105 $cm^{3}$\\ 
               
 \textbf{Step 3}. Mass of the Al foil = Volume $\times$ Density\\
                                    = 105 $\times$ 2.7\\
                                    = 283.5 grams\\ 
                                    
 \textbf{Step 4}. If 27grams $\rightarrow$ 1mole then 283.5grams $\rightarrow$ ? moles\\
                so, $\frac{283.5}{27}$ = 10.5 moles\\ 
            
                
\textbf{Step 5}. If 283.5grams $\rightarrow$ 2500cm then 27grams $\rightarrow$ ? cm\\
                So, $\frac{(27 \times 2500)}{283.5}$ = 238.09 cm $\approx$ 2.4 metre\\
                
\end{minipage}} \\ \\ 

                  
\textbf{So, to get 1 mole of Al atoms we have to roll 2.4 meters of Al foil of given dimensions.} \\ \\ \\ \\
%===================================================================


%===================================================================                               
\question
\label{Q2:Dollar bills}

How much each of 2 SGD, 10 USD, and Rs.50 bills weigh? How many atoms are there in each of the bills?(What fraction of a mole?).\\
\textbf{Answer}: \\ 
\textbf{Step 1}. Required information\\ 

                 1. SGD notes are made of Biaxially oriented polypropylene (PP)($C_{3}H_{6}$)$_{n}$. Which is mainly Carbon and Hydrogen atoms.\\
                 
                 2. USD and Indian Rupees are made of Paper which is 75\% Cotton and 25\% Linen. Cotton contains 91\% cellulose ($C_{6}H_{10}O_{5}$)$_{n}$.\\ 
                 
                 3. Dimensions of bills in (width $\times$ length $\times$ thickness)in cm \\
                 SGD 2 = (6.3 $\times$ 12.6 $\times$ 0.01) \\
                 USD 10 = (6.63 $\times$ 15.6 $\times$ 0.01)\\
                 INR 50 = (7.3 $\times$ 14.7 $\times$ 0.01)\\   
                 
                 4. Density of Polypropylene = 0.93 grams/$cm^{3}$.              \\Density of Cellulose = 1.5grams/$cm^{3}$.\\
                 
                 5. Molecular weight of Polypropylene($C_{3}H_{6}$)= 12$\times$3 + 1$\times$6 = 42 grams/mole. Which means 1 mole of PP weighs 42 grams. \\
                 
                 Molecular weight of Cellulose ($C_{6}H_{10}O_{5}$)= 12$\times$6 + 1$\times$10 + 16$\times$5 = 162 grams/mole. Which means, 1 mole of Cellulose weighs 162 grams. \\ \\


\fbox{\begin{minipage}{42em}  

\textbf{Step 2}. 1. SGD 2\\

 (1.1) Volume = (6.3 $\times$ 12.6 $\times$ 0.01)$\approx 0.8cm^{3}$.\\ 
 
 (1.2) Mass = Density $\times$ Volume \\
                   = 0.93 grams/$cm^{3}$ $\times$ 0.8$cm^{3}$ \\
                   = 0.76 grams. \\
                                                             
 (1.3) If 42 grams of PP $\rightarrow$ 1 mole of PP molecules then 0.76 grams $\rightarrow$ ? molecules\\ 
                                          
                                          
 So, $\frac{0.76 grams}{42 grams}$ $\times$ (6 $\times$ $10^{23}$) $\approx$ \textbf{1.1 $\times$ $10^{21}$ of PP molecules in 1 SGD 2 note bill.} And to have 1 mole of PP molecules, we should have 55 notes (=SGD 110). \\ \\
                                         
                                         
                                         
2.USD 10 \\ 

(2.1) Volume = ($6.63 \times 15.6 \times 0.01) \approx 1cm^{3}$.\\ 

(2.2) Mass    = Density$\times$ Volume \\
	             = $1.5 grams/cm^{3}\times 1cm^{3}$ \\
                     = 1.5 grams  (But actual weight of the note is $\approx$ 1 gram, due to the reduced density of polymer chain after packaging.)\\ 
                      
 (2.3) If 162 grams of Cellulose $\rightarrow$ 1 mole of Cellulose molecules then 1 grams $\rightarrow$ ? molecules\\ 
                     
So, {$\frac{1 gram} {162 grams}$} $\times$  $6\times10^{23}$  $\approx$   \textbf{$3.7\times10^{21}$} of Cellulose molecules in 1 USD 10 note bill  and to have 1 mole of Cellulose molecules, we should have 162 notes (= USD 1620). \\ \\

3.INR 50 \\ 
(3.1) Volume = ($7.3 \times 14.7\times 0.01) \approx 1.1cm^{3}$.\\       
(3.2) Mass    = Density $\times$ Volume \\
                     = 1.5 grams/$cm^{3}$ $\times$ 1$cm^{3}$ \\
                     = 1.7 grams  (Similar to the USD, actual weight of the note is $\approx$ 1 gram due to the reduced density of polymer chain after packaging. )\\   
                     
(3.3) If 162 grams of Cellulose $\rightarrow$ 1 mole of Cellulose molecules then 1 grams $\rightarrow$ ? molecules\\ 

 So, $\frac{ 1 gram }{162 grams}$ $\times$ (6 $\times$ $10^{23}$) $\approx$ \textbf{3.7 $\times$ $10^{21}$ of Cellulose molecules in 1 INR 50 note bill.} And to have 1 mole of Cellulose molecules, we should have 162 notes (=INR 8100). \\ \\  

\end{minipage}}\\   \\ \\ \\ \\ \\               

%=================================================================== 
  
  
%===================================================================                   
\question
\label{Q3:Hair}

How many hair are there on the scalp of average adult human? How big a animal has to be to have 1 mole of hair on his body? \\
\textbf{Answer}: \\
\textbf{Step 1}: Required information.\\
                1. For humans, average body surface area (BSA)  = $\frac{\sqrt{(W \times H)}}
                   {60}$.\\
                   Where, W = weight in Kg and H = height in centimetres.\\
                   For average adult human with weight = 70Kg and height = 180 cm, \\
                   BSA = $\frac{\sqrt{70\times180}}{60}$ $\approx$ $2m^{2}$.\\
                   
                2. Scalp surface area $\approx$ $500cm^{2}$.\\
                   No. of hair follicles per$cm^{2}$ $\approx$ 100. And on an average, each 
                  follicle will have 2 hair. So, there are $200 hair/cm^{2}$.\\ 
                   
\fbox{\begin{minipage}{42em}  
         \textbf{Step 2}: 1. If 1$cm^{2}$ $\rightarrow$ 200 hair, then 500$cm^{2}$ $\rightarrow$ 
                             ? hair.\\
                             So, $\frac{500cm^{2}\times200}{1cm^{2}}$ = 100,000 hair on human 
                             scalp. \\
                          
                          2. Considering same density of hair (200 hair/$cm^{2}$) over all the 
                             body surface area, a human with 2$m^{2}$ of body surface area 
                             will have \\
                      $\frac{200\times2\times10^{4}cm^{2}}{1cm^{2}}$ = 4$\times10^{6}$ hair.\\
                 
                          3. Now, if 4 million hair per 2$m^{2}$ of body surface area then 
                             6$\times10^{23}$ hair $\rightarrow$ ? body surface area.\\
                             So, $\frac{6\times10^{23}\times 2m^{2}}{4\times10^{6}}$ $\approx$ 
                             $3\times10^{19}m^{2}$ = $3\times10^{12} km^{2}$.\\ 
                  
                          4. The Sun is the celestial body with largest surface area in our 
                             solar system.\\ Radius of the Sun = 695,700 Km.$\approx$ 700,000 
                             Km \\
                             Area of the Sun = $\pi$ $\times$ $r^{2}$ \\
                                             = $\pi$ $\times(7\times10^{5})^{2}$   \\
                                             $\approx$ $1.5\times10^{12}Km^{2}$. 
\end{minipage}}\\  \\
         \textbf{So, an animal has to be as big as twice the size of the Sun to have 1 mole of hair.} \\                  
%=================================================================== 


%=================================================================== 
\question
\label{Q4:A4 paper}

How many sheets of A4 size paper collectively has 1 mole of carbon atoms?\\
\textbf{Answer}: \\ 
\textbf{Step 1}. Required information\\ 
1. From given data, A paper of area $1m^{2}$ (=$10^{4}cm^{2}$) weighs 80 grams. \\ 
2. Size of A4 paper = 21cm $\times$ 30cm \\ 
3. Molecular weight of Carbon = 12 grams/mole.
 Area of A4 paper = 21cm $\times$ 30cm = 630 $cm^{2}$\\ 

\fbox{\begin{minipage}{38em} 

\textbf{Step 2}. If $10^{4}cm^{2}$ $\rightarrow$ 80grams then $630cm^{2}$ $\rightarrow$ ? grams.\\ 
So, $\frac{(630 \times 80)}{10^4}$ = 5.04 grams. \\
Now, If 5 grams $\rightarrow$ 1 A4 paper then 12 grams $\rightarrow$ ? papers .
So, $\frac{12}{5}$ = 2.4 meters. 
\end{minipage}} \\ \\
\textbf{So, nearly 2 and a half sheets of A4 paper have 1 mole of carbon atoms.} \\ \\

%=================================================================== 


%=================================================================== 
\question
\label{Q5:Styrofoam}

How big is 1 cubic volume of 1 mole of Styrofoam?\\
\textbf{Answer}: \\
\textbf{Step 1}. Required information \\
Styrofoam is made of Polystyrene ($C_{8}H_{8}$)$_{n}$. \\
Polystyrene density = 0.96-1.04 gram/ $cm^{3}$.\\
Styrofoam weighs = 0.05 gram/$cm^{3}$ \\
Molecular weight of Polystyrene($C_{8}H_{8}$) = 12 $\times$ 8 + 1 $\times$ 8\\ = 104 grams/mole.\\ 

\fbox{\begin{minipage}{38em}
 
\textbf{Step 2}. If 0.05 grams $\rightarrow$ 1$cm^{3}$ then 104 grams $\rightarrow$ ? $cm^{3}$.\\

So, $\frac{104}{0.05} = 2080 cm^{3}$. \\ 
\end{minipage}} \\ 

\textbf{1 mole of Styrofoam has 2080$cm^{3}$ volume. Which is a cube of side 12.76 cm.} \\
%=================================================================== 


%=================================================================== 
\question
\label{Q6:Diamond}

How big is a diamond of 1 mole of carbon?\\
\textbf{Answer}: \\
\textbf{Step 1}. Required information. \\
                 Molecular weight of Carbon = 12 grams/mole. \\
                 Mass of diamond $\approx$ 12 grams. \\
                 Density of diamond = 3.515 grams/$cm^{3}$ \\ 
                                  
\fbox{\begin{minipage}{38em}   
               
\textbf{Step 2}.  1. Volume of a diamond of mass 12 grams \\
                     $Volume = \frac{mass}{Density}$ \\
                     $ Volume = \frac{12}{3.5} = 3.4 cm^{3}$ \\
                     
                  2. Volume of a sphere = $ \frac{4}{3} \times \Pi \times r^{3}$ \\
                    $\Longrightarrow$ 3.4 $cm^{3}$ = $ \frac{4}{3} \times \Pi \times r^{3}$ \\
                    $\Longrightarrow$ $\frac{10.2}{12.56}$ = $r^{3}$ \\
                    $\Longrightarrow$ $0.82 cm^{3}$ = $r^{3}$ \\
                    $\Longrightarrow$ r = 0.9 cm \\

 \end{minipage}} \\
 
 \textbf{So, A diamond of 1 mole of Carbon will be as big as a sphere of diameter 1.8 cm.} \\ 
%=================================================================== 

                    
%=================================================================== 
\question
\label{Q7: A dot of ink}

How many moles of molecules are there in a dot of white board marker ink? \\
\textbf{Answer}: \\
\textbf{Step 1}. Required information. \\
A white board marker pen ink mainly contains Ethanol($C_{2}H_{6}O$) or Isopropanol($C_{3}H_{8}O).$ \\
Density of Ethanol = 0.789 g/$cm^{3}$ $\approx$ 0.8 g/$cm^{3}$\\
Density of Isopropanol = 0.786 g/$cm^{3}$ $\approx$ 0.8g/$cm^{3}$\\
Considering marker pen tip = 5mm in diameter, a dot on whiteboard is a very tiny disc of thickness 0.1mm\\
Molecular weight of Ethanol ($C_{2}H_{6}O$) = 12$\times$2 + 1$\times$6 + 16$\times$1 \\
 = $46 grams/mole.$ \\ 

\fbox{\begin{minipage}{38em} 
                                           
\textbf{Step 2}. 
1. Volume of a disk = $\Pi \times r^{2} \times thickness$ \\
                               = $(3.14 \times (0.25)^{2} \times 0.001 ) cm^{3} $ \\
                              $\approx 2\times10^{-4} cm^{3}$ \\

2. Mass	= Volume $\times$ Density \\
 		= $2 \times 10^{-4} \times 0.8$ \\
    		= 1.6 $\times 10^{-4} $grams \\ 

3. So, if 46 grams of Ethanol $\rightarrow$ 1 mole of Ethanol molecules then $1.6 \times10^{-4}$  grams $\rightarrow$ ? molecules of Ethanol.\\

So, $\frac{(1.6\times10^{-4}) \times (6\times10^{23})}{46} \approx 2\times10^{18} molecules$. \\  
\end{minipage}} \\  

\textbf{So, a $5mm$ diameter dot of marker pen ink has $2\times10^{18}$ molecules.}  \\
                       
%===================================================================                       
 
 
%=================================================================== 
\question
\label{Q8:Phone battery}

How many moles of electrons does a fully charged phone battery store? \\
\textbf{Answer}: \\
\textbf{Step 1}. Required information. \\

1. Phone battery current rating is in Ampere-Hour(AH). Today's smart phones generally have 3AH of current rating.Which means a fully charged 3AH battery can produce 3Amp current for 1 hour or 1 Amp current for 3 hours. \\

2. 1 Coulomb = 1 Ampere.Second \\
              
3. 1 electron has $1.6 \times 10^{-19}$ coulombs of charge.   
              
 So, 1 Coulomb has $\frac{1}{1.6 \times10^{-19}}$ = $6.25 \times 10^{18}$ electrons. \\   
                                               
\fbox{\begin{minipage}{38em} 

\textbf{Step 2}. If 1 Amp current $\rightarrow 6.25 \times 10^{18}$ electrons, then 3 Amp $\rightarrow$ ? electrons. \\

So, $\frac{(3 \times 6.25\times10^{18})} {1} = 1.875 \times 10^{19}$ electrons.  \\ 

\textbf{So, a fully charged phone battery stores $\approx 1.9 \times 10^{19}$ electrons.}\\ \\

\textbf{Step 3}. Counting further, if $6.25 \times 10^{18} $ electrons $\rightarrow 1 Amp$ current, then $ 6\times10^{23} $electrons $\rightarrow$ ? Amp current. \\ 

            $\frac{6\times 10^{23}} {6.25 \times 10^{18}} \approx 10^{5} $Amp Current. \\
\end{minipage}} \\ 

\textbf{So, 1 mole of electrons will produce $10^{5} Amp$ currents.}   \\ \\           
%===================================================================              
                

%=================================================================== 
\question
\label{Q9:Lightest morning}

Why we weigh lightest in the morning after the sleep? Where does this lost mass go? \\
\textbf{Answer}: \\
\textbf{Step 1}. Required information.\\
               1. During sleep human inhales $\approx$ 500 ml of air per breathe. And take 16 breathes per minute. So, he inhales $\approx$ 7-8 liters of air per minute. \\
               2. Average sleep duration is 8 hours.  \\ 
               
\fbox{\begin{minipage}{38em}                
\textbf{Step 2}. 1. Per breathe intake of air is 500 ml. Out of which $\frac({1}{5}^{th})$ is Oxygen $\approx$ 100 ml. \\ 
                 This inhaled Oxygen combines with carbon and becomes $CO_{2}$ which is then exhaled.This exhaled $CO_{2}$ is around $\frac({1}{5})^{th}$ of the inhaled Oxygen. Which is $\approx$ 20ml. \\

\textbf{Step 3}. 1.Exhaled $CO_{2}$ carries 0.012 grams of carbon in each breathe. \\ 
        
If per minute 16 breathes, then in 8 hours total 7680 breathes.\\
So, 0.012 grams$\times$ 7680 breathes $\approx$ 92 grams of mass is lost in the form of carbon during 8 hours of sleep.\\ Also, some mass is lost in the form of sweat. Hence we weigh lightest in the morning.
\end{minipage}} \\ \\
%=================================================================== 


%=================================================================== 
\question
\label{Q10:ATPs}
How many moles of ATP molecules does each cell in a human body synthesize? How many ATP molecules are required to run for 1 hour? And how many ATP molecules are required to run at a speed of 60Km/hr? \\
\textbf{Answer}: \\
\textbf{Step 1}: Required information. \\
                 Average ATPs required per day by humans $\approx$ 200-300 moles. Let's consider 240 moles of ATPs a day. Which is $\approx$ 10 moles of ATPs per hour. \\ 

\fbox{\begin{minipage}{38em} 


\textbf{Step 2}: Total number of cells in human body $\approx 4 \times10^{12}$ \\
                
$\Rightarrow 4\times 10^{12}$ cells synthesize 240 moles of ATPs/day.\\

If, $4\times10^{12}$ cells $\Longrightarrow$ produce 240 moles of ATPs/day, then 1 cell $\Longrightarrow$ ? ATPs a day? \\

$\frac{240\times6\times10^{23}} {4\times10^12}$ = $36\times 10^{11}$ ATP molecules.\\  

\end{minipage}} \\

\textbf{So, $36\times 10^{11}$ ATP molecules by each cell per day. } \\ \\ \\ \\ \\ \\ \\ \\ \\ \\
%=================================================================== 


%=================================================================== 
\question
\label{Q11:Moles of water molecules}

How many moles of water molecules does a human consume in a day? And in his life span? \\
\textbf{Answer}: \\
\textbf{Step 1}. Required information. \\

1. 1 mole of water molecules = 18 grams of water. \\
                           1 litre water = 1000 grams of water. \\
1 litre of water contains $\frac{1000}{18}$ $\approx$ 55 moles. \\ 

\fbox{\begin{minipage}{38em}

\textbf{Step 2}. On an average, daily water intake for humans 2 litres. \\
So, 2 litres $\times$ 55 moles = 110 moles/day. \\

\textbf{Step 3}. Number of days per year = 365. \\
 Average life span of humans = 60 years. \\
So, for 60 $times$ 365 = 21900 days, total water consumption is 21900 $\times$ 110 moles $\approx 2.4\times10^{6}$ moles of water. Which is 43362 litres of water.

\end{minipage}} \\ 

\textbf{So, A human consumes 110 moles of water a day and 2.4 million moles in his life span.}                \\ \\
%=================================================================== 


%=================================================================== 
\question
\label{Q12:E.coli}
For a bacteria E. coli, how big a ball of 1 mole of E. coli will be? \\
\textbf{Answer}:\\
\textbf{Step 1}: Required information. \\
                E.coli is a cylindrical shape bacteria with a radius of 0.5$\mu$m and a length of about 2$\mu$m. \\ \\
\fbox{\begin{minipage}{38em}
\textbf{Step 2}: 1. Volume of single E. coli cell = $\pi$ $\times r^{2} \times length$ \\
                                                  = $3.14 \times 0.5^{2} \times 2$ \\
                                                  $\approx$ 1.6$\mu$ $m^{3}$\\   \\
                 2. If 1 E.coli $\rightarrow$ 1.6$\mu$ $m^{3}$ of volume, then 1 mole of E.coli $\rightarrow$ ? volume. \\
                So, $\frac{(1.6 \times (10^{-6})^{3} \times 6 \times 10^{23})}{1}$ $\approx$ $9.6 \times 10^{5} m^{3}$ \\ \\
                 3. Volume of a sphere = $\frac{4}{3}$ $\times \pi \times r^{3}$ \\
                    $9.6 \times 10^{5} m^{3}$ = $\frac{4}{3}$ $\times \pi \times r^{3}$ \\
                    $\Longrightarrow r^{3}$ $\approx$ 62 metres. \\ 
\end{minipage}} \\ \\
                 \textbf{1 mole of E.coli will make a ball of diameter 124m.}\\ \\ \\ \\ \\ \\ \\ \\ \\

%=================================================================== 


%=================================================================== 
\question
\label{Q13:E.coli and DNA basepairs}
How big a packed ball of E. coli will be, so that there are 1 mole of DNA base pairs? \\
\textbf{Answer}: \\
\textbf{Step 1}. Required information.\\
                 E.coli genome consists of a single molecule of DNA containing $\approx$ 4.7 million base pairs. \\ 
                 
\fbox{\begin{minipage}{42em}
\textbf{Step 2}. 1. If $4.7\times10^{6}$ base pairs $\rightarrow$ in 1 E.coli cell, then $6\times10^{23}$ base pairs $\rightarrow$ ? E.coli cells. \\

                 $\frac{(6\times10^{23})}{(4.7\times10^{6})}$ = $1.27 \times 10^{17}$ E.coli cells.\\ So, $\approx$ $1.3\times10^{17}$ E.coli cells will have 1 mole of DNA base pairs. \\ \\ 
                 2. Now, if 1 E.coli cell $\rightarrow$ 1.6$\mu$ $m^{3}$ of volume, then $1.3\times10^{17}$ E.coli cells $\rightarrow$ ?volume. \\ 
                 
                 So, $\frac{1.3\times10^{17}\times1.6\times10^{-18}}{1}$ $\approx$ 0.2 $m^{3}$.\\ \\
                 3. Volume of a sphere = $\frac{4}{3}$ $\times \pi \times r^{3}$ \\
                           0.2 $m^{3}$ = $\frac{4}{3}$ $\times \pi \times r^{3}$ \\
                    $\Longrightarrow$ r $\approx$ 0.37 m = 37cm. \\
\end{minipage}} \\ 

                  \textbf{So, a packed ball of E.coli cells containing 1 mole DNA base pairs will be of size 74cm in diameter.}   \\ \\
%=================================================================== 


%=================================================================== 
\question
\label{Q14:E.coli and water}
How big a ball of E. coli will be to have 1 mole of water? \\
\textbf{Answer}. \\
\textbf{Step 1}. Required information \\
                1. E.coli bacteria contains 70$\%$ water. \\
                2. 1 mole of E.coli has $9.6\times10^{5}m^{3}$ of volume. And 70$\%$ of                 $9.6\times10^{5}m^{3}$ $\approx$ $6.7\times10^{5}m^{3}$. \\
                \\
\fbox{\begin{minipage}{42em}
\textbf{Step 2}. 1.Density of water = 1000Kg/$m^{3}$.\\
                 If 1$m^{3}$ $\rightarrow$ 1000Kg then $6.7\times10^{5}m^{3}$ $\rightarrow$ ? Kg. \\
                 So, $\frac{6.7\times10^{5}}{1000}$ = $6.7\times10^{8}$Kg = $6.7\times10^{11}$grams of water.\\ \\
                 2. Now, if $6.7\times10^{11}$grams $\rightarrow$ $6.7\times10^{5}m^{3}$ volume, then 18 grams $\rightarrow$ ? volume.\\ \\
                 So, $\frac{18\times6.7\times10^{5}}{6.7\times10^{11}}$ = $18\times10^{-6}m^{3}$ = 18$cm^{3}$. \\ \\
                 3. Volume of sphere = $\frac{4}{3}$ $\pi \times r^{3}$ \\
                         18$cm^{3}$ = $\frac{4}{3}$ $\pi \times r^{3}$ \\
                         $\Longrightarrow$ r = 1.65 cm \\ 
\end{minipage}} \\ \\
                 \textbf{So, a ball of E.coli of diameter 3cm will hold 1 mole of water.} \\ \\ \\ \\ \\ 
%=================================================================== 


%=================================================================== 
%\question
%\label{Q15:Leaves on trees}
%For a randomly selected tree, how many leaves per tree are there? Which type of trees will have higher leaf density- trees with smaller leaves or trees with bigger leaves? 
%%=================================================================== 
%
%
%%=================================================================== 
%\question
%\label{Q16:Number of letters}
%Has humankind printed 1 mole of letters till now? If not then how much more time is required to print 1 mole of letters?
%%=================================================================== 
%
%
%%=================================================================== 
%\question
%\label{Q17:Memory units}
%How many memory units does a human being has in his life?
%=================================================================== 


%=================================================================== 
%\question
%\label{Q4}



\end{questions}
%=================================================================== 
%=================================================================== 
 
\end{document} 