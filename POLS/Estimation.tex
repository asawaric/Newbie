%===================================================================
% Latex file: 11.dec.2016,Exam style:
%===================================================================
%===================================================================
% By default LaTeX uses large margins.  This doesn't work well on exams; problems
% end up in the "middle" of the page, reducing the amount of space for students
% to work on them.
% For an exam, single spacing is most appropriate
%===================================================================

\documentclass[11pt]{exam}
\RequirePackage{amssymb, amsfonts, amsmath, latexsym, verbatim, xspace, setspace,mathpazo}
\usepackage[margin=1in]{geometry}
\usepackage{textcomp}
\singlespacing

%===================================================================

\begin{document}
\title{\textbf{Estimation : Interesting questions and Back of the envelope calculations}}
\maketitle

%===================================================================
\subsection{Avogadro Number: A Mole.}
%===================================================================

\begin{questions}

%===================================================================
\question
\label{Q1:Al Foil}

What is the length of a sheet of household Aluminium foil that has 1 mole of Aluminium atoms?\\
\textbf{Answer }: \\
\textbf{Step 1}. Required information.\\
Atomic mass of Aluminium = 27. This means 1 mole of Aluminium(Al) atoms weigh 27 grams.\\
Density of Al = 2.7 grams/cm$^{3}$\\
Dimensions of household Al foil:\\ Length = 25 meter(2500cm), Width = 30cm, and Thickness = 14 microns(0.0014cm).\\

%\begin{align*}                
%Atomic \,mass\, of\, Aluminium &= 27.\, This\, means\, 1\, mole\, of\, Aluminium(Al)\, atoms\, weigh\, 27\, grams.\\
%Density\, of\, Al &= 2.7 grams/cm^{3}\\
%Dimensions\, of\, household\, Al\, foil&:\\ Length = 25 meter(2500cm), Width = 30cm, and Thickness = 14 microns(0.0014cm).
%\end{align*}
		                                    
\fbox{\begin{minipage}{28em} 
                 
\textbf{Step 2}. Volume of Al foil for above dimensions.
\begin{align*} 
Volume &= Width \times Length \times Thickness\\
       &= (30 \times 0.014 \times 2500) cm^{3}\\
       &= 105 cm^{3}.         
\end{align*}                 
        
\textbf{Step 3}. Mass of the Al foil 
\begin{align*} 
Mass &= Volume \times Density\\
     &= 105 \times 2.7\\
     &= 283.5 grams.
\end{align*}  
                
\textbf{Step 4}. If 27grams $\rightarrow$ 1mole then 283.5grams $\rightarrow$ ? moles.
                 \begin{align*}
                  so,\, \frac{283.5}{27} &= 10.5 moles.
                 \end{align*} 
                
\textbf{Step 5}. If 283.5grams $\rightarrow$ 2500cm then 27grams $\rightarrow$ ? cm.
                \begin{align*}                
                So,\, \frac{(27 \times 2500)}{283.5} = 238.09 cm \approx 2.4 metre.
                \end{align*}
\end{minipage}} \\ \\ 

                  
\textbf{So, to get 1 mole of Al atoms we have to roll 2.4 meters of Al foil of given dimensions.} \\ \\ \\ \\
%===================================================================


%===================================================================                               
\question
\label{Q2:Dollar bills}

How much each of 2 SGD, 10 USD, and Rs.50 bills weigh? How many atoms are there in each of the bills?(What fraction of a mole?).\\
\textbf{Answer}: \\ 
\textbf{Step 1}. Required information\\ 

                 1. SGD notes are made of Biaxially oriented polypropylene (PP)($C_{3}H_{6}$)$_{n}$. Which is mainly Carbon and Hydrogen atoms.\\
                 
                 2. USD and Indian Rupees are made of Paper which is 75\% Cotton and 25\% Linen. Cotton contains 91\% cellulose ($C_{6}H_{10}O_{5}$)$_{n}$.\\ 
                 
                 3. Dimensions of bills in (width $\times$ length $\times$ thickness)in cm.
                 \begin{align*}
                 SGD\, 2 &= (6.3 \times 12.6 \times 0.01) \\
                 USD\, 10 &= (6.63 \times 15.6 \times 0.01)\\
                 INR\, 50 &= (7.3 \times 14.7 \times 0.01)   
                 \end{align*}                
                            
                 4. Density
                    \begin{align*} Density\, of\, Polypropylene &= 0.93\, grams/cm^{3}.\\              
                    Density\, of\, Cellulose &= 1.5\,grams/cm^{3}.
                    \end{align*}
                    
                 
                 5. Molecular weight of Polypropylene($C_{3}H_{6}$)= 12$\times$3 + 1$\times$6 = 42 grams/mole. Which means 1 mole of PP weighs 42 grams. \\
                 
                 Molecular weight of Cellulose ($C_{6}H_{10}O_{5}$)= 12$\times$6 + 1$\times$10 + 16$\times$5 = 162 grams/mole. Which means, 1 mole of Cellulose weighs 162 grams. \\ \\


\fbox{\begin{minipage}{42em}  

\textbf{Step 2}. 1. SGD 2\\

 (1.1) Volume = (6.3 $\times$ 12.6 $\times$ 0.01)$\approx 0.8cm^{3}$.\\ 
 
 (1.2) Mass = Density $\times$ Volume \\
                   = 0.93 grams/$cm^{3}$ $\times$ 0.8$cm^{3}$ \\
                   = 0.76 grams. \\
                                                             
 (1.3) If 42 grams of PP $\rightarrow$ 1 mole of PP molecules then 0.76 grams $\rightarrow$ ? molecules\\ 
                                          
                                          
 So, $\frac{0.76 grams}{42 grams}$ $\times$ (6 $\times$ $10^{23}$) $\approx$ \textbf{1.1 $\times$ $10^{21}$ of PP molecules in 1 SGD 2 note bill.} And to have 1 mole of PP molecules, we should have 55 notes (=SGD 110). \\ \\
                                         
                                         
                                         
2.USD 10 \\ 

(2.1) Volume = ($6.63 \times 15.6 \times 0.01) \approx 1cm^{3}$.\\ 

(2.2) Mass    = Density$\times$ Volume \\
	             = $1.5 grams/cm^{3}\times 1cm^{3}$ \\
                     = 1.5 grams  (But actual weight of the note is $\approx$ 1 gram, due to the reduced density of polymer chain after packaging.)\\ 
                      
 (2.3) If 162 grams of Cellulose $\rightarrow$ 1 mole of Cellulose molecules then 1 grams $\rightarrow$ ? molecules\\ 
                     
So, {$\frac{1 gram} {162 grams}$} $\times$  $6\times10^{23}$  $\approx$   \textbf{$3.7\times10^{21}$} of Cellulose molecules in 1 USD 10 note bill  and to have 1 mole of Cellulose molecules, we should have 162 notes (= USD 1620). \\ \\

3.INR 50 \\ 
(3.1) Volume = ($7.3 \times 14.7\times 0.01) \approx 1.1cm^{3}$.\\       
(3.2) Mass    = Density $\times$ Volume \\
                     = 1.5 grams/$cm^{3}$ $\times$ 1$cm^{3}$ \\
                     = 1.7 grams  (Similar to the USD, actual weight of the note is $\approx$ 1 gram due to the reduced density of polymer chain after packaging. )\\   
                     
(3.3) If 162 grams of Cellulose $\rightarrow$ 1 mole of Cellulose molecules then 1 grams $\rightarrow$ ? molecules\\ 

 So, $\frac{ 1 gram }{162 grams}$ $\times$ (6 $\times$ $10^{23}$) $\approx$ \textbf{3.7 $\times$ $10^{21}$ of Cellulose molecules in 1 INR 50 note bill.} And to have 1 mole of Cellulose molecules, we should have 162 notes (=INR 8100). \\ \\  

\end{minipage}}\\   \\ \\ \\ \\ \\               

%=================================================================== 
  
  
%===================================================================                   
\question
\label{Q3:Hair}

How many hair are there on the scalp of average adult human? How big a animal has to be to have 1 mole of hair on his body? \\
\textbf{Answer}: \\
\textbf{Step 1}: Required information.\\
                1. For humans, average body surface area (BSA)  = $\frac{\sqrt{(W \times H)}}
                   {60}$.\\
                   Where, W = weight in kg and H = height in centimetres.\\
                   For average adult human with weight = 70 kg and height = 180 cm, \\
                   BSA = $\frac{\sqrt{70\times180}}{60}$ $\approx$ $2m^{2}$.\\
                   
                2. Scalp surface area $\approx$ $500cm^{2}$.\\
                   No. of hair follicles per$cm^{2}$ $\approx$ 100. And on an average, each 
                  follicle will have 2 hair. So, there are $200 hair/cm^{2}$.\\ 
                   
\fbox{\begin{minipage}{42em}  
         \textbf{Step 2}: 1. If 1$cm^{2}$ $\rightarrow$ 200 hair, then 500$cm^{2}$ $\rightarrow$ 
                             ? hair.\\
                             So, $\frac{500cm^{2}\times200}{1cm^{2}}$ = 100,000 hair on human 
                             scalp. \\
                          
                          2. Considering same density of hair (200 hair/$cm^{2}$) over all the 
                             body surface area, a human with 2$m^{2}$ of body surface area 
                             will have \\
                      $\frac{200\times2\times10^{4}cm^{2}}{1cm^{2}}$ = 4$\times10^{6}$ hair.\\
                 
                          3. Now, if 4 million hair per 2$m^{2}$ of body surface area then 
                             6$\times10^{23}$ hair $\rightarrow$ ? body surface area.\\
                             So, $\frac{6\times10^{23}\times 2m^{2}}{4\times10^{6}}$ $\approx$ 
                             $3\times10^{19}m^{2}$ = $3\times10^{12} km^{2}$.\\ 
                  
                          4. The Sun is the celestial body with largest surface area in our 
                             solar system.\\ Radius of the Sun = 695,700 km.$\approx$ 700,000 
                             km \\
                             Area of the Sun = $\pi$ $\times$ $r^{2}$ \\
                                             = $\pi$ $\times(7\times10^{5})^{2}$   \\
                                             $\approx$ $1.5\times10^{12}km^{2}$. 
\end{minipage}}\\  \\
         \textbf{So, an animal has to be as big as twice the size of the Sun to have 1 mole of hair.} \\                  
%=================================================================== 


%=================================================================== 
\question
\label{Q4:A4 paper}

How many sheets of A4 size paper collectively has 1 mole of carbon atoms?\\
\textbf{Answer}: \\ 
\textbf{Step 1}. Required information\\ 
1. From given data, A paper of area $1m^{2}$ (=$10^{4}cm^{2}$) weighs 80 grams. \\ 
2. Size of A4 paper = 21cm $\times$ 30cm \\ 
3. Molecular weight of Carbon = 12 grams/mole.
 Area of A4 paper = 21cm $\times$ 30cm = 630 $cm^{2}$\\ 

\fbox{\begin{minipage}{38em} 

\textbf{Step 2}. If $10^{4}cm^{2}$ $\rightarrow$ 80grams then $630cm^{2}$ $\rightarrow$ ? grams.\\ 
So, $\frac{(630 \times 80)}{10^4}$ = 5.04 grams. \\
Now, If 5 grams $\rightarrow$ 1 A4 paper then 12 grams $\rightarrow$ ? papers .
So, $\frac{12}{5}$ = 2.4 meters. 
\end{minipage}} \\ \\
\textbf{So, nearly 2 and a half sheets of A4 paper have 1 mole of carbon atoms.} \\ \\

%=================================================================== 


%=================================================================== 
\question
\label{Q5:Styrofoam}

How big is 1 cubic volume of 1 mole of Styrofoam?\\
\textbf{Answer}: \\
\textbf{Step 1}. Required information \\
Styrofoam is made of Polystyrene ($C_{8}H_{8}$)$_{n}$. \\
Polystyrene density = 0.96-1.04 gram/ $cm^{3}$.\\
Styrofoam weighs = 0.05 gram/$cm^{3}$ \\
Molecular weight of Polystyrene($C_{8}H_{8}$) = 12 $\times$ 8 + 1 $\times$ 8\\ = 104 grams/mole.\\ 

\fbox{\begin{minipage}{38em}
 
\textbf{Step 2}. If 0.05 grams $\rightarrow$ 1$cm^{3}$ then 104 grams $\rightarrow$ ? $cm^{3}$.\\

So, $\frac{104}{0.05} = 2080 cm^{3}$. \\ 
\end{minipage}} \\ 

\textbf{1 mole of Styrofoam has 2080$cm^{3}$ volume. Which is a cube of side 12.76 cm.} \\
%=================================================================== 


%=================================================================== 
\question
\label{Q6:Diamond}

How big is a diamond of 1 mole of carbon?\\
\textbf{Answer}: \\
\textbf{Step 1}. Required information. \\
                 Molecular weight of Carbon = 12 grams/mole. \\
                 Mass of diamond $\approx$ 12 grams. \\
                 Density of diamond = 3.515 grams/$cm^{3}$ \\ 
                                  
\fbox{\begin{minipage}{38em}   
               
\textbf{Step 2}.  1. Volume of a diamond of mass 12 grams \\
                     $Volume = \frac{mass}{Density}$ \\
                     $ Volume = \frac{12}{3.5} = 3.4 cm^{3}$ \\
                     
                  2. Volume of a sphere = $ \frac{4}{3} \times \Pi \times r^{3}$ \\
                    $\Longrightarrow$ 3.4 $cm^{3}$ = $ \frac{4}{3} \times \Pi \times r^{3}$ \\
                    $\Longrightarrow$ $\frac{10.2}{12.56}$ = $r^{3}$ \\
                    $\Longrightarrow$ $0.82 cm^{3}$ = $r^{3}$ \\
                    $\Longrightarrow$ r = 0.9 cm \\

 \end{minipage}} \\
 
 \textbf{So, A diamond of 1 mole of Carbon will be as big as a sphere of diameter 1.8 cm.} \\ 
%=================================================================== 

                    
%=================================================================== 
\question
\label{Q7: A dot of ink}

How many moles of molecules are there in a dot of white board marker ink? \\
\textbf{Answer}: \\
\textbf{Step 1}. Required information. \\
A white board marker pen ink mainly contains Ethanol($C_{2}H_{6}O$) or Isopropanol($C_{3}H_{8}O).$ \\
Density of Ethanol = 0.789 g/$cm^{3}$ $\approx$ 0.8 g/$cm^{3}$\\
Density of Isopropanol = 0.786 g/$cm^{3}$ $\approx$ 0.8g/$cm^{3}$\\
Considering marker pen tip = 5mm in diameter, a dot on whiteboard is a very tiny disc of thickness 0.1mm\\
Molecular weight of Ethanol ($C_{2}H_{6}O$) = 12$\times$2 + 1$\times$6 + 16$\times$1 \\
 = $46 grams/mole.$ \\ 

\fbox{\begin{minipage}{38em} 
                                           
\textbf{Step 2}. 
1. Volume of a disk = $\Pi \times r^{2} \times thickness$ \\
                               = $(3.14 \times (0.25)^{2} \times 0.01 ) cm^{3} $ \\
                              $\approx 2\times10^{-3} cm^{3}$ \\

2. Mass	= Volume $\times$ Density \\
 		= $2 \times 10^{-3} \times 0.8$ \\
    		= 1.6 $\times 10^{-3} $grams \\ 

3. So, if 46 grams of Ethanol $\rightarrow$ 1 mole of Ethanol molecules then $1.6 \times10^{-3}$  grams $\rightarrow$ ? molecules of Ethanol.\\

So, $\frac{(1.6\times10^{-3}) \times (6\times10^{23})}{46} \approx 2\times10^{19} molecules$. \\  
\end{minipage}} \\  

\textbf{So, a $5mm$ diameter dot of marker pen ink has $2\times10^{19}$ molecules.}  \\
                       
%===================================================================                       
 
 
%=================================================================== 
\question
\label{Q8:Phone battery}

How many electrons does a fully charged phone battery store? \\
\textbf{Answer}: \\
\textbf{Step 1}. Required information. \\

1. Phone battery current rating is in Ampere-Hour(AH). Today's smart phones generally have 3AH of current rating.Which means a fully charged 3AH battery can produce 3Amp current for 1 hour or 1 Amp current for 3 hours. \\

2. 1 Coulomb = 1 Ampere.Second \\
              
3. 1 electron has $1.6 \times 10^{-19}$ coulombs of charge.   
              
 So, 1 coulomb has $\frac{1}{1.6 \times10^{-19}}$ = $6.25 \times 10^{18}$ electrons. \\   
                                               
\fbox{\begin{minipage}{38em} 

\textbf{Step 2}. 1 Amp current is 1 coulomb of electrons per second.\\

                 So, for 1 Amp current there are $6.25 \times 10^{18}$ electrons in 1 second. And $6.25 \times 10^{18} \times 3600 $ = $2.25\times 10^{22}$ electrons in 1 hour. \\ 
                 If 1 Amp-hour = $2.25\times 10^{22}$ electrons, then 3 Amp-hour $\rightarrow$ ? electrons. \\

So, $\frac{(3 \times 2.25\times10^{22})} {1} = 6.75 \times 10^{22}$ electrons.  \\ 
\end{minipage}} \\ 

\textbf{So, a fully charged phone battery stores $\approx 6.75 \times 10^{22}$ electrons.}\\ \\

%\textbf{Step 3}. Counting further, if $6.25 \times 10^{18} $ electrons $\rightarrow 1 Amp$ current, then $ 6\times10^{23} $electrons $\rightarrow$ ? Amp current. \\ 
%
%            $\frac{6\times 10^{23}} {6.25 \times 10^{18}} \approx 10^{5} $Amp Current. \\
%\end{minipage}} \\ 
%
%\textbf{So, 1 mole of electrons will produce $10^{5} Amp$ currents.}   \\ \\           
%===================================================================              
                

%=================================================================== 
\question
\label{Q9:Lightest morning}

Why we weigh lightest in the morning after the sleep? Where does this lost mass go? \\
\textbf{Answer}: \\
\textbf{Step 1}. Required information.\\
               1. During sleep human inhales $\approx$ 500 ml of air per breathe. And take 16 breathes per minute. So, he inhales $\approx$ 7-8 liters of air per minute. \\
               2. Average sleep duration is 8 hours.  \\ 
               
\fbox{\begin{minipage}{38em}                
\textbf{Step 2}. 1. Per breathe intake of air is 500 ml. Out of which $\frac{1}{5}$$^{th}$ is Oxygen $\approx$ 100 ml. \\ 
                 This inhaled Oxygen combines with carbon and becomes $CO_{2}$ which is then exhaled.This exhaled $CO_{2}$ is around $\frac{1}{5}$$^{th}$ of the inhaled Oxygen. Which is $\approx$ 20ml. \\

\textbf{Step 3}. 1.Exhaled $CO_{2}$ carries 0.012 grams of carbon in each breathe. \\ 
        
If per minute 16 breathes, then in 8 hours total 7680 breathes.\\
So, 0.012 grams$\times$ 7680 breathes $\approx$ 92 grams of mass is lost in the form of carbon during 8 hours of sleep.\\ Also, some mass is lost in the form of sweat. Hence we weigh lightest in the morning.
\end{minipage}} \\ \\
%=================================================================== 


%=================================================================== 
\question
\label{Q10:ATPs}
How many moles of ATP molecules does each cell in a human body synthesize? How many ATP molecules are required to run for 1 hour? And how many ATP molecules are required to run at a speed of 60Km/hr? \\
\textbf{Answer}: \\
\textbf{Step 1}: Required information. \\
                 Average ATPs required per day by humans $\approx$ 200-300 moles. Let's consider 240 moles of ATPs a day. Which is $\approx$ 10 moles of ATPs per hour. \\ 

\fbox{\begin{minipage}{38em} 


\textbf{Step 2}: Total number of cells in human body $\approx 4 \times10^{12}$ \\
                
$\Rightarrow 4\times 10^{12}$ cells synthesize 240 moles of ATPs/day.\\

If, $4\times10^{12}$ cells $\Longrightarrow$ produce 240 moles of ATPs/day, then 1 cell $\Longrightarrow$ ? ATPs a day? \\

$\frac{240\times6\times10^{23}} {4\times10^12}$ = $36\times 10^{11}$ ATP molecules.\\  

\end{minipage}} \\

\textbf{So, $36\times 10^{11}$ ATP molecules by each cell per day. } \\ \\ \\ \\ \\ \\ \\ \\ \\ \\
%=================================================================== 


%=================================================================== 
\question
\label{Q11:Moles of water molecules}

How many moles of water molecules does a human consume in a day? And in his life span? \\
\textbf{Answer}: \\
\textbf{Step 1}. Required information. \\

1. 1 mole of water molecules = 18 grams of water. \\
                           1 litre water = 1000 grams of water. \\
1 litre of water contains $\frac{1000}{18}$ $\approx$ 55 moles. \\ 

\fbox{\begin{minipage}{38em}

\textbf{Step 2}. On an average, daily water intake for humans 2 litres. \\
So, 2 litres $\times$ 55 moles = 110 moles/day. \\

\textbf{Step 3}. Number of days per year = 365. \\
 Average life span of humans = 60 years. \\
So, for 60 $times$ 365 = 21900 days, total water consumption is 21900 $\times$ 110 moles $\approx 2.4\times10^{6}$ moles of water. Which is 43362 litres of water.

\end{minipage}} \\ 

\textbf{So, A human consumes 110 moles of water a day and 2.4 million moles in his life span.}                \\ \\
%=================================================================== 


%=================================================================== 
\question
\label{Q12:E.coli}
For a bacteria E. coli, how big a ball of 1 mole of E. coli will be? \\
\textbf{Answer}:\\
\textbf{Step 1}: Required information. \\
                E.coli is a cylindrical shape bacteria with a radius of 0.5$\mu$m and a length of about 2$\mu$m. \\ \\
\fbox{\begin{minipage}{38em}
\textbf{Step 2}: 1. Volume of single E. coli cell = $\pi$ $\times r^{2} \times length$ \\
                                                  = $3.14 \times 0.5^{2} \times 2$ \\
                                                  $\approx$ 1.6$\mu$ $m^{3}$\\   \\
                 2. If 1 E.coli $\rightarrow$ 1.6$\mu$ $m^{3}$ of volume, then 1 mole of E.coli $\rightarrow$ ? volume. \\
                So, $\frac{(1.6 \times (10^{-6})^{3} \times 6 \times 10^{23})}{1}$ $\approx$ $9.6 \times 10^{5} m^{3}$ \\ \\
                 3. Volume of a sphere = $\frac{4}{3}$ $\times \pi \times r^{3}$ \\
                    $9.6 \times 10^{5} m^{3}$ = $\frac{4}{3}$ $\times \pi \times r^{3}$ \\
                    $\Longrightarrow r^{3}$ $\approx$ 62 metres. \\ 
\end{minipage}} \\ \\
                 \textbf{1 mole of E.coli will make a ball of diameter 124m.}\\ \\ \\ \\ \\ \\ \\ \\ \\

%=================================================================== 


%=================================================================== 
\question
\label{Q13:E.coli and DNA basepairs}
How big a packed ball of E. coli will be, so that there are 1 mole of DNA base pairs? \\
\textbf{Answer}: \\
\textbf{Step 1}. Required information.\\
                 E.coli genome consists of a single molecule of DNA containing $\approx$ 4.7 million base pairs. \\ 
                 
\fbox{\begin{minipage}{42em}
\textbf{Step 2}. 1. If $4.7\times10^{6}$ base pairs $\rightarrow$ in 1 E.coli cell, then $6\times10^{23}$ base pairs $\rightarrow$ ? E.coli cells. \\

                 $\frac{(6\times10^{23})}{(4.7\times10^{6})}$ = $1.27 \times 10^{17}$ E.coli cells.\\ So, $\approx$ $1.3\times10^{17}$ E.coli cells will have 1 mole of DNA base pairs. \\ \\ 
                 2. Now, if 1 E.coli cell $\rightarrow$ 1.6$\mu$ $m^{3}$ of volume, then $1.3\times10^{17}$ E.coli cells $\rightarrow$ ?volume. \\ 
                 
                 So, $\frac{1.3\times10^{17}\times1.6\times10^{-18}}{1}$ $\approx$ 0.2 $m^{3}$.\\ \\
                 3. Volume of a sphere = $\frac{4}{3}$ $\times \pi \times r^{3}$ \\
                           0.2 $m^{3}$ = $\frac{4}{3}$ $\times \pi \times r^{3}$ \\
                    $\Longrightarrow$ r $\approx$ 0.37 m = 37cm. \\
\end{minipage}} \\ 

                  \textbf{So, a packed ball of E.coli cells containing 1 mole DNA base pairs will be of size 74cm in diameter.}   \\ \\
%=================================================================== 


%=================================================================== 
\question
\label{Q14:E.coli and water}
How big a ball of E. coli will be to have 1 mole of water? \\
\textbf{Answer}. \\
\textbf{Step 1}. Required information \\
                1. E.coli bacteria contains 70$\%$ water. \\
                2. 1 mole of E.coli has $9.6\times10^{5}m^{3}$ of volume. And 70$\%$ of                 $9.6\times10^{5}m^{3}$ $\approx$ $6.7\times10^{5}m^{3}$. \\
                \\
\fbox{\begin{minipage}{42em}
\textbf{Step 2}. 1.Density of water = 1000kg/$m^{3}$.\\
                 If 1$m^{3}$ $\rightarrow$ 1000kg then $6.7\times10^{5}m^{3}$ $\rightarrow$ ? kg. \\
                 So, $\frac{6.7\times10^{5}}{1000}$ = $6.7\times10^{8}$kg = $6.7\times10^{11}$grams of water.\\ \\
                 2. Now, if $6.7\times10^{11}$grams $\rightarrow$ $6.7\times10^{5}m^{3}$ volume, then 18 grams $\rightarrow$ ? volume.\\ \\
                 So, $\frac{18\times6.7\times10^{5}}{6.7\times10^{11}}$ = $18\times10^{-6}m^{3}$ = 18$cm^{3}$. \\ \\
                 3. Volume of sphere = $\frac{4}{3}$ $\pi \times r^{3}$ \\
                         18$cm^{3}$ = $\frac{4}{3}$ $\pi \times r^{3}$ \\
                         $\Longrightarrow$ r = 1.65 cm \\ 
\end{minipage}} \\ \\
                 \textbf{So, a ball of E.coli of diameter 3cm will hold 1 mole of water.} \\ \\ \\ \\ \\ \\ \\ \\
%=================================================================== 


%===================================================================                 
\question
\label{Q15:Cold water to loose 100Kcal/day}
How much iced water(temp = 0\textdegree{}C) is required to drink to lose 100 Kilo calories a day? \\
\textbf{Answer}. \\
\textbf{Step 1}. Required information \\
                1. Specific heat of water = 1 cal/gram.\textdegree{}C \\
\\
\fbox{\begin{minipage}{42em}
\textbf{Step 2}. Heat = mass $\times$ specific heat $\times$ temperature change. \\
                 100 Kcal = mass $\times$ 1 cal/gram.\textdegree{}C $\times$ (37-0)\\
                 10$^{5}$ cal = mass $\times$ 1 cal/gram.\textdegree{}C $\times$ 37.\\
                 $\Longrightarrow$ mass = $\frac{10^{5} cal}{1 cal/gram.\textdegree{}C \times 37}$ \\
                 mass of the water = 2702.7 grams $\approx$ 2.7 litres or more.\\


\end{minipage}} \\ \\
                 \textbf{So, to lose 100 Kcal a day one need to drink atleast 2.7 litres of iced water.} \\ \\ 
%=================================================================== 


%=================================================================== 
\question
\label{Q16:Cold water weight loss}
How much energy a person will lose if he sits in cold water(temp = 4\textdegree{}C) for 1 hour? \\
\textbf{Answer}. \\
\textbf{Step 1}. Required information \\
                1. Normal body temperature of a person = 37\textdegree{}C\\
                2. Cold water temperature = max 21-22\textdegree{}C \\
                3. Temperature of cold water from fridge = 4\textdegree{}C \\
                4. Specific heat of water = 1 cal/gram.\textdegree{}C \\
                5. Mass of a person = 70 Kg = 70000 grams.\\
                
\fbox{\begin{minipage}{42em}
\textbf{Step 2}. 1. Heat = mass $\times$ specific heat $\times$ temperature change. \\ \\
                         = 70000 $\times$ 1 cal/gram.\textdegree{}C $\times$ (37-4)\textdegree{}C \\
                         = 2.31 $\times$ $10^{6}$ $\times$ 1 \\
                         = 2310 Kcal \\ \\
                 2.For cold water of temperature 21\textdegree{}C, \\
                   Heat = 70000 $\times$ 1 cal/gram.\textdegree{}C $\times$ (37-21)\textdegree{}C \\
                         = 1120 Kcal \\ \\
                % 3.How many calories are lost in 1 hr?
                         
                       

\end{minipage}} \\ \\
                 \textbf{The speed of Google man is 1.25 metres/second.} \\ \\ \\
%%=================================================================== 



%%=================================================================== 

%\question
%\label{Q17:Calories required to walk a marathon}
%How many calories are required to walk a marathon of 70 kilometres at 80 metres/minute? \\
%\textbf{Answer}. \\
%\textbf{Step 1}. Required information \\
%                 1. 
%\\
%\fbox{\begin{minipage}{42em}
%\textbf{Step 2}. 1. 
%                         
%                       
%
%\end{minipage}} \\ \\
%                 \textbf{The speed of Google man is 1.25 metres/second.} \\ \\ \\
%%=================================================================== 


%%=================================================================== 
\question
\label{Q18:Calories required to chew a raisin}
What is the time required to chew a single raisin before you lose the energy? \\ as much as raisin gives you?
\textbf{Answer}. \\
\textbf{Step 1}. Required information \\
                 1. Calorie of a single raisin =  2 kcal. \\
                 2. Mass of mouth = m = 400 grams = 0.4 kg   \\    
                 3. Height to which one needs to lift his mouth for chewing = h =  1 cm = 0.01 metre.   \\
                 4. Acceleration due to gravity = g = 10 metres/$sec^{2}$    
\\
\fbox{\begin{minipage}{42em}
\textbf{Step 2}. 1. Energy utilized in chewing (up and down) = 2 $\times$ mgh \\
                                                             = 2 $\times$ 0.4 $\times$ 10 $\times$0.01 \\
                                                             = 0.08 joules \\ \\                                        
                 2. 2 kcal = 8.36 kjoules. \\
                    If 0.08 joules $\rightarrow$ in 1 chewing \\
                    then 8.36 kjoules $\rightarrow$  in ? chewing? \\
                    $\Longrightarrow$ $\frac{8.36 \times 10^{3}}{0.08}$                                   = 104.5 $\times$ $10^{3}$chewings. \\ \\
                    
                 3. If 1 chewing $\rightarrow$ 1 second \\
                    then 104.5 $\times$ $10^{3}$ $\rightarrow$ ? seconds \\
                    $\Longrightarrow$ $\frac{104000}{3600}$ = 29 hours = 1.2 days. \\
                   
\end{minipage}} \\ \\
                 \textbf{Hence, to chew a single raisin before you loose the energy as much as raisin gives will require continuous chewing for 1.2 day} \\ \\ 
%%=================================================================== 


%%=================================================================== 
\question
\label{Q19:Speed of a Google man}
How fast a google man walks? \\
\textbf{Answer}. \\
\textbf{Step 1}. Required information \\
                 1. Speed given in Google map = 3 Kilometres in 40 minutes\\
\\
\fbox{\begin{minipage}{42em}
\textbf{Step 2}. 1. 40 minutes = 2400 seconds. \\ \\
                 2. If in 2400 seconds $\rightarrow$ 3000 metres, \\ then in 1 second $\rightarrow$ ? metres. \\ 
                    So, in 1 second $\frac{3000 metres}{2400 seconds}$ = 1.25 metres per second.\\
   
\end{minipage}} \\ \\
                 \textbf{The speed of Google man is 1.25 metres/second.} \\ \\ \\ \\ \\ \\ \\ \\ \\ \\ \\ \\ \\ \\ \\ 
%%=================================================================== 


%%=================================================================== 
\question
\label{Q20:Black body radiation}
How much energy do we lose to our surroundings due to the temperature between our body and our surroundings? \\
1. Assuming Sun as a black body radiator, how much energy does the Sun pull out per second?\\
2. How much of this energy does the Earth receive? \\
3. Assuming that there is an object of human body temperature in the space, how much power does it radiate? \\
4. If the radiator in part 3 above is also bathe by another black body radiator that is at room temperature, then what is the net power loss by object in part 3? \\ \\
\textbf{Answer}. \\ 
\textbf{Step 1}. Required information \\
                1. Normal human body temperature = 37\textdegree{}C  = 311 K\\
                2. Room temperature = 27\textdegree{}C = 300 K \\
                3. Temperature of the Sun (Photosphere) = 5800 K. \\
                4. Stefan-Boltzmann's constant = 5.67 $\times$ 10$^{-8}$ watts/m$^{2}$.K$^{4}$ \\
                5. Radius of the Sun = 7 $\times$ 10$^{8}$ metres.\\
                6. Distance between the Sun and the Earth = 1 AU = 149 $\times$ 10$^{9}$ metres. \\
                7. Surface area of the Sun = 4 $\times$ $\pi$ $\times$  $R_{sun}^{2}$   \\
                                           = 8.8 $\times$ 10$^{9}$ m$^{2}$\\  \\
\fbox{\begin{minipage}{42em}
\textbf{Step 2}. 1. Power radiated by the Sun as black body radiator = j$^{*}$ = $\sigma \times T^{4}$. \\
                    = 5.67 $\times$ 10$^{-8}$ $\times$ (5800)$^{4}$.\\
          \textbf{ j$^{*}$  = 6.4 $\times$ 10$^{7}$ watts/m$^{2}$.}\\ \\
%----------------------------------------------------------------------------------------------           
                 2. Total power radiated from the surface area of the Sun = $\sigma \times T^{4}$ $\times$ Surface area of the Sun \\
                    = $\sigma$ $\times$ $T^{4}$ $\times$ 4 $\times$ $\pi$ $\times$  $R_{sun}^{2}$    \\
                    = 6.4 $\times$ 10$^{7}$ watts/m$^{2}$ $\times$ 8.8 $\times$ 10$^{9}$ m$^{2}$ \\
\textbf{ Total power radiated by the Sun  = 3.94 $\times$ 10$^{26}$ watts. =  3.94 $\times$ 10$^{26}$ joules/sec} \\ \\
%--------------------------------------------------------------------------------------------       
                  3. \textbf{How much of this energy does Earth receive?} \\
                     Solar irradiance = $S_{o}$ = It is the power per unit area received  from the Sun in the form of electromagnetic radiation. %in the wavelength range of the measuring instrument. \\
                     \\ 
                     Solar irradiance on an object at a distance D away from the Sun is found by dividing the total power emitted from the Sun by the surface area over which the Sunlight falls. that is,\\
%$S_{o}$ = $\frac{Total power radiated by the Sun}{Surface area over which sunlight falls}$ \\        
$S_{o}$ = $\frac{\sigma \times T^{4} \times 4 \times \pi \times  R_{sun}^{2}}{4 \times \pi \times (D_{SuntoEarth}^{2})}$ \\ \\
       = $\frac{3.94 \times 10^{26}watts}{1.9 \times 10^{12}m^{2}}$\\ \\
                     = 1430 watts/m$^{2}$ \\ \\
%--------------------------------------------------------------------------------------------                     
                     Power received by the Earth from Sunlight =  $S_{o}$ $\times$ $\pi$ $\times$ $R_{Earth}^{2}$ \\
                     = 1430 watts/m$^{2}$ $\times$ 3.14 $\times$ $(6371 \times 10^{3})^{2}m^{2}$ \\
                   \textbf{ Power received by the Earth from the Sunlight = 1.9 $\times$ 10$^{17}$ watts.} \\ \\
%---------------------------------------------------------------------------------------------                   
                   4.\textbf {Assuming that there is an object of human body temperature (HBT) in the space, how much power does it radiate?} \\
                    Temperature of the object = 311 K \\
                    $j^{*}$ = $\sigma$ $\times$ $T^{4}$ \\
                            = 5.67 $\times$ 10$^{-8}$ watts/m$^{2}$.K$^{4}$ $\times$ (311 K)$^{4}$ \\
                   \textbf{$j^{*}_{HBT}$= 530.4 watts/m$^{2}$} \\ \\
%--------------------------------------------------------------------------------------------                   
                   5. If the radiator in part 3 above is also bathe by another black body radiator that is at room temperature (RT), then what is the net power loss by object in part 3? \\
         $j^{*}$ = $\sigma$ $\times$ $T^{4}$ \\
                 = 5.67 $\times$ 10$^{-8}$ watts/m$^{2}$.K$^{4}$ $\times$ (300 K)$^{4}$ \\ \textbf{$j^{*}_{RT}$= 459.3 watts/m$^{2}$} \\ \\ 
%---------------------------------------------------------------------------------------------                 
                   6. The net radiative power loss = $Q_{1-2}$ = A1 $\times$ $\sigma$ $\times$ (($T_{1}$)$^{4}$ - ($T_{2}$)$^{4}$ ) \\
                                                   = 2 m$^{2}$ $\times$ 5.67 $\times$ 10$^{-8}$ watts/m$^{2}$.K$^{4}$ $\times$ (311$^{4}$ - 300$^{4}$) \\
                      \textbf{The net radiative power loss = 136 watts = 136 joules/sec  $\Rightarrow$ 11.56 $\times$ 10$^{6}$ joules/day = 2764.8 kcal/day} \\ \\                      
%--------------------------------------------------------------------------------------
                   7. The net radiative power loss for surrounding temperature at 0\textdegree{}C = 273 K.
                   The net radiative power loss = $Q_{1-2}$ = A1 $\times$ $\sigma$ $\times$ (($T_{1}$)$^{4}$ - ($T_{2}$)$^{4}$ ) \\
                                                   = 2 m$^{2}$ $\times$ 5.67 $\times$ 10$^{-8}$ watts/m$^{2}$.K$^{4}$ $\times$ (311$^{4}$ - 273$^{4}$) \\
                                                   = 430 watts.
                                                   
                      \textbf{The net radiative power loss = 430 watts = 430 joules/sec  = 37 $\times$ 10$^{6}$ joules/day = 8879 kcal/day.}
                      
                                       
\end{minipage}} \\ \\ \\ \\
%%=================================================================== 


%%=================================================================== 
\question
\label{Q21: Power consumption per day by Singapore}
How much power Singapore consumes per day? \\ 
1. Most of this energy comes from Natural-gas mainly Methane ($CH_{4}$).What is the natural energy density of methane? \\
2. How many swimming pools full of LNG (Liquefied natural gas) Singapore burns a day? \\
3. What is the mass of $CO_{2}$ emitted per day? \\
4. The amount of $CO_{2}$ emitted per day is equivalent to how many trees? \\
5. How long would it take for Singapore to get covered with trees to compensate $CO_{2}$ emitted per day? \\ \\


\textbf{Answer}. \\
\textbf{Step 1}. Required information \\
                1. Density of LNG = 450 kg/$m^{3}$. \\
                2. Energy density of LNG = 0.45 kg/litres = 50 MJ/kg.\\
                3. Volume of a Olympic size swimming pool = 2500 $m^{3}$ = 2.5 $\times$ 10$^{6}$ litres.\\
                4. Daily energy consumption of energy in Singapore = 450 $\times$ 10$^{12}$ joules.\\
                5. Specific carbon content in Natural gas = 0.75 $kg_{C}$/$kg_{fuel}$
\\
\fbox{\begin{minipage}{42em}
\textbf{Step 2}. 1. How many swimming pools full of LNG (Liquefied natural gas) Singapore burns a day? \\     Energy density of LNG = 50 MJ/kg. \\
                    Daily energy consumption of energy in Singapore = 450 $\times$ 10$^{12}$ joules \\ \\
                    If 50 MJ in 1 kg then 450$\times$ 10$^{12}$ joules $\rightarrow$ ? kg. \\
                    $\Rightarrow$ $\frac{450 \times 10^{12} joules \times 1 kg}{50\times10^6 joules}$ = 9$\times$ 10$^{6}$ kg. \\ \\
                    Density of LNG = 450 kg/$m^{3}$.\\
                    If 1$m^{3}$ = 450 kg \\
                    then 2500$m^{3}$ $\rightarrow$ ? kg. \\ \\
                    $\Rightarrow$ $\frac{2500 m^{3} \times 450 kg}{1m^{3}}$ = 1125 $\times$ 10$^{3}$ kg\\ \\
                    Now, if 1125 $\times$ 10$^{3}$ kg in 1 swimming pool, \\   
                    then 9$\times$ 10$^{6}$ kg in $\rightarrow$ ? swimming pools \\
                    $\Rightarrow$ $\frac{9\times 10^{6} kg}{1125 \times 10^{3}kg}$  = 8 swimming pools.\\ \\
                    \textbf{So, Singapore needs 8 Olympic size swimming pools full of LNG for the requirement of 1 day energy consumption.} \\ \\
                    
                    2. What is the mass of $CO_{2}$ emitted per day? \\
                       To calculate the $CO_{2}$ emission from a fuel, the carbon content of the fuel must be multiplied with the ratio of molecular weight of the $CO_{2}$ to the molecular weight of C = $\frac{44}{12}$ = 3.7\\ \\
                       So, $CO_{2}$ emission from natural gas = 0.75 $\times$ 3.7 = 2.775 $kg_{C}$/$kg_{fuel}$. \\
                       
                       \textbf{For, $CO_{2}$ emission from 9$\times$ 10$^{6}$ kg of LNG = 2.775$kg_{C}$/$kg_{fuel}$ $\times$ 9$\times$ 10$^{6}$ kg = 25.2 $\times$ 10$^{6}$ $kg_{C}$} \\ \\
                    3. The amount of $CO_{2}$ emitted per day is equivalent to how many trees? \\
                      1 tree consumes 21 kg of $CO_{2}$ in 1 year. \\
                      If in 365 days, 21 kg of $CO_{2}$ then in 1 day $\rightarrow$ ? kg of $CO_{2}$ \\ \\
                      $\frac{21 kg}{365}$ =0.057 kg/day. \\ \\
                      If 0.057kg by 1 tree then 25 $\times$ 10$^{6}$ kg by $\rightarrow$ ? trees. \\
                      $\frac{25 \times 10^{6}}{0.057}$ = 435 million trees.\\
                      
                      The amount of $CO_{2}$ emitted per day is equivalent to 435 million trees. \\ \\

\end{minipage}} \\ \\                                                         
                     
\fbox{\begin{minipage}{42em}                                     
                    4. How long would it take for Singapore to get covered with trees to compensate $CO_{2}$ emitted per day? \\ 
                       For Mahogany trees, $CO_{2}$ density = 300 $\times$ 10$^{3}$ kg/hectare = 30 $\times$ 10$^{6}$ kg/ km$^{2}$ \\ 
                    
                    If 30 $\times$ 10$^{6}$kg of $CO_{2}$ in 1 km$^{2}$ \\
                    then 25 $\times$ 10$^{6}$ $\rightarrow$ ? km$^{2}$ \\ \\
                    $\Rightarrow$ $\frac{25 \times 10^{6} \times 1km^{2}}{30 \times 10^{6}kg}$ = 0.83 km$^{2}$ \\ \\
                    If 0.83 km$^{2}$ area of Mahogany trees are = 1 day of $CO_{2}$ emission\\
                    then 719 km$^{2}$ $\rightarrow$ ? days of $CO_{2}$ emission \\ \\
                    $\Rightarrow$ $\frac{719 km^{2} \times  1 day}{0.83 km^{2}}$ = 862.8 days =2.36 years. \\ \\
                    \textbf{So, Total area of Singapore will be covered in 2.36 years.}                    
                 
  
\end{minipage}} \\ \\ \\ \\ \\
%%=================================================================== 


%%=================================================================== 
\question
\label{Q22:Swimming pool full of AA batteries}
How many swimming pools full of AA batteries can fulfil the energy requirement of 1 day by Singapore? \\
How many AA batteries will need to make a layer of AA batteries over the entire area of Singapore? \\
\textbf{Step 1}. Required information \\
                1. Dimensions of AA (Alkaline) batteries : \\
                   Length = 5 cm \\ Diameter = 1.5cm \\ weight = 30 grams \\
                2. Maximum energy at nominal voltage and 50 mA drain = 3 watt-hour = 10.8 $\times$ 10$^{3}$ joules. \\ \\
                
\fbox{\begin{minipage}{42em}
\textbf{Step 2}. 1. Consumption of energy/day in Singapore = 450 $\times$ 10$^{12}$ joules. \\  \\
                    If 10.8 $\times$ 10$^{3}$ joules in 1 AA battery,\\
                    then 450 $\times$ 10$^{12}$ joules $\rightarrow$ ? AA batteries. \\ \\
                    $\Rightarrow$ $\frac{450 \times 10^{12}}{10.8 \times 10^{3}}$ = 42 $\times$ 10$^{9}$ AA batteries.\\ \\
                    \textbf{So, Singapore's 1 day requirement of energy can be stored in 42 billion AA batteries.} \\ \\
                 2. 1 AA battery weighs $\approx$ 30 grams. \\
                    then 42 billion batteries will weigh $\rightarrow$ ? grams. \\ \\
                   $\frac{42 \times 10^{9} \times 30 grams}{1}$ = 126 $\times$ 10$^{10}$ grams = 126 $\times$ 10$^{7}$ kg.  \\ \\
                 3. Capacity of 1 Olympic size swimming pool = (Volume) = 2500 $m^{2}$.
                    Energy density of of AA Alkaline batteries = 1.3MJ/Liter. \\
                    If 1 liter $\rightarrow$ 1.3 MJ \\
                    then 1000 liters $rightarrow$ ? MJ \\ \\
                    $\Rightarrow$ $\frac{1000 \times 1.3 MJ}{1}$ = 1300 MJ/ 1000 liters = 1300 MJ / m$^{3}$. \\ \\
                 4. Specific energy of AA Alkaline batteries = 0.67 MJ/kg. \\
                    If 0.67 MJ per 1 kg \\
                    then 1300 MJ per ? kg \\ \\
                    $\Rightarrow$ $\frac{1300 MJ \times 1 kg}{0.67 MJ}$ = 1940 kg $\Rightarrow$ 1940 kg/ m$^{3}$. \\ \\
                    Volume of 1 swimming pool = 2500 m$^{3}$ \\
                    If 1 m$^{3}$ holds 1940 kg of AA batteries \\ 
                    then 2500 m$^{3}$ $\rightarrow$ holds ? kg batteries. \\ \\
                    $\Rightarrow$ $\frac{2500 m^{3}\times 1940 kg}{1 m^{3}}$ = 4.85 $\times$ 10$^{6}$ kg. \\ \\
                    If 30 grams $\rightarrow$ 1 AA battery, \\
                    then 4.85 $\times$ 10$^{6}$ kg $\rightarrow$ ? batteries. \\ \\
                     $\Rightarrow$ $\frac{4.85\times 10^{9} g}{30 g}$ $\approx$ 162 million batteries. \\ \\ 
                    If 162 million batteries stored in 1 swimming pool \\
                    then 42 billion batteries are stored in $\rightarrow$ ? swimming pools . \\ \\
                    $\Rightarrow$ $\frac{42 \times 10^{9}}{162 \times 10^{6}}$ = 259 swimming  pools. \\ \\
\end{minipage}} \\ \\                    

\fbox{\begin{minipage}{42em}                    
                  5. Area of Olympic size swimming pool = 1250 m$^{2}$. \\
                      If 1250 m$^{2}$ =  1 swimming pool \\
                      then 719 $\times$ 10$^{6}$ m$^{2}$ = ? swimming pools. \\ \\
                      $\Rightarrow$ $\frac{719 \times 10^{6}}{1250}$ = 575 $\times$ 10$^{3}$ swimming pools. \\ \\
                  6. If 259 Swimming pools can store energy required for 1 day \\
                     then 575 $\times$ 10$^{3}$ swimming pools can store $\rightarrow$ energy required for ? day's. \\ \\
                     $\Rightarrow$ $\frac{575 \times 10^{3}}{259}$ = 2220 days $\approx$ 6 years. \\ \\
                   \textbf{So, Total area of Singapore as a swimming pool full of AA batteries will hold the energy requirement of Singapore for 6 years.} \\ \\
                  7. How many AA batteries will need to make a layer of AA batteries over the entire area of Singapore? \\ \\
                  Length of AA battery = 5 cm = 0.05 m \\
                  Diameter of AA battery = 1.5 cm = 0.015 m \\
                  Area of AA battery cell = Area of a cylinder = 2 $\times \pi \times r \times (r + h)$ \\
                                          = 2 $\times 3.14 \times 0.015 \times (0.015 + 0.05)$ \\ 
                   \textbf{Area of a AA battery cell $\approx$ 6 $\times$ 10$^{-3}$ m$^{2}$ } \\ \\ 
                   If 6 $\times$ 10$^{-3}$ m$^{2}$ = 1 AA battery \\
                   then 719 $\times$ 10$^{6}$ m$^{2}$ $\rightarrow$ ? batteries. \\ \\
                   $\Rightarrow$ $\frac{719 \times 10^{6} m^{2}}{6 \times 10^{-3} m^{2}}$ = 117 billion AA batteries. \\ \\
                   \textbf{So, Total cover area of Singapore with a sheet, 117 billion AA batteries are required. These number of batteries can suffice the energy requirement of 3 days for Singapore.}
                     
                       
\end{minipage}} \\ \\
%%=================================================================== 


%%=================================================================== 
\question
\label{Q23: Rainwater catchment and water consumption by Singapore}
1. How much rainwater is stored in Singapore per year? \\ 
2. How many litres of water is consumed in Singapore per day? This is equal to how many     swimming pools of water?\\
3. What is the total volume of Singapore's all the reservoirs?

\textbf{Step 1}. Required information \\
                 In Singapore, \\
                 1. Average number of rainy days = 178. \\
                 2. Average rainfall per year = 2400 mm/year. \\
                 3. Rainfall volume per square meters of area  = 1.725 $\times$ 10$^{9}$ m$^{3}$/719 $\times$ 10$^{6}$ m$^{2}$.\\ 
                 4. Per day water demand = 430 million gallons = 1.627 $\times$ 10$^{9}$ litres/day. $\Rightarrow$ approx 6 $\times$ 10$^{11}$ litres/year. 
                \\ \\
                
\fbox{\begin{minipage}{42em}
\textbf{Step 2}. 1. How much rainwater is stored in Singapore per year? What is the total volume of Singapore's all the reservoirs? \\ 
                    "The average annual rainfall in Singapore = 2400 mm. About 50\% of the land area is used for water catchment. \\ Available water based on an average of 2200 mm of annual rainfall gives 12 $\times$ 10$^{8}$ m$^{3}$ of water falling over the mainland of Singapore. About half of this would be lost through evaporation and transpiration. \\ Rough estimates give the \textbf{total available water from existing catchment area to be about 1.6 $\times$ 10$^{8}$ m$^{3}$/year}. \\ \\
                    There are total 17 reservoirs in Singapore. The \textbf{total maximum storage capacity of these reservoirs $\approx$ 108 $\times$ 10$^{6}$ m$^{3}$(= 108 billion litres).}" \textit{(Ref from the Book: 'The coastal Environmental Profile of Singapore' by Lin Sien Chia, Habibullah Khan (Ph. D.), L. M. Chou)}. \\ \\
                 2. How many litres of water is consumed in Singapore per day? This is equal to how many swimming pools of water? \\
                 Water consumption in Singapore $\approx$ 1.6 $\times$ 10$^{9}$ litres/day. \\ \\
                 
                 Volume of an Olympic size swimming pool = 2500 m$^{3}$ = 2.5 $\times$ 10$^{6}$ litres. \\ 
                 If  2.5 $\times$ 10$^{6}$ litres in 1 swimming pool\\
                 then 1.6 $\times$ 10$^{9}$ litres $\rightarrow$ ? swimming pools. \\ \\
                 $\Rightarrow$ $\frac{1.6 \times 10^{9}}{2.5 \times 10^{6}}$ = 640 Swimming pools. \\ \\
                 \textbf{So, Singapore needs water of 640 Swimming pools per day and 240,000 Swimming pools per year.}
                 
\end{minipage}} \\ \\                    
%%=================================================================== 


%%=================================================================== 
\question
\label{Q24:Energy consumption and C$O_{2}$ emission of cars running on different fuels.}
What is the energy consumption per kilometre by cars running on CNG, Petrol, Diesel and Electricity? How much C$O_{2}$ is emitted by each of these cars running on different fuels? \\
\textbf{Answer}. \\
\textbf{Step 1}. Required information \\
                 1. Energy content of the engine fuels: \\
                    Gasoline = 9.5 kwh/litre \\
                    Diesel = 9.7 kwh/litre \\
                    CNG = 53.6 MJ/kg = 14.8 kwh/kg\\ %(or 7.5 kwh/litre) \\
                    %Biodiesel = 9.2 kwh/litre \\
                    Electric car = 23 kwh/100km = 8.8 kwh/litre\\
                    \\
                 2. Mileage (km/litre) of cars: \\
                    Gasoline car = 20 \\
                    Diesel car =  30  \\
                    CNG car  =   25 km/kg  \\
                    Electric car = 4.3 km/kwh = 38.46 km/litre  \\
                    %Biodiesel car =    \\ 
                    \\
                 3. C$O_{2}$ emission by different fuel cars ($kg_{c}$/kwh): \\
                    Gasoline car =  0.24 \\
                    Diesel car =  0.27  \\
                    CNG car =  0.0545 $kg_{c}$/scf $\rightarrow$  0.197 $kg_{c}$/kwh \\ %= 2.92$kg_{c}$/kg   \\
                   % Biodiesel =    0.27  \\
                    Electric car =  0.527  \\
                 
                 %For 2012 Honda civic natural gas car, average miles per gallon equivalent  = 31 MPG$_{e}$ = 32 gal/100 miles.\\
                 %2. Energy per gallon of CNG = 28.81 KWh/3.78 lit = 104 MJ/ 3.78 lit. $\Rightarrow$ 27 MJ/lit .
                 %31 MPG$_{e}$ $\approx$ 13 km/lit $\Rightarrow$ 7.6 lit/100 km. \\
%                    \\ \\
\fbox{\begin{minipage}{42em}
\textbf{Step 2}.    1. Gasoline car. \\
                       Goes average 20 km/litre. \\
                       Consumes 9.5 kwh/litre of energy.\\
                       Emits 240 grams/kwh $\rightarrow$ 2.28 kg/litre. \\
                       On an average, for 20,000 km of distance travelled a year,(needs 1000 litres of Gasoline) car running on Gasoline emits 2.28 tonnes of carbon dioxide in a year. \\ \\
                    2. Diesel car. \\
                       Goes average 30 km/litre. \\
                       Consumes 9.7 kwh/litre of energy.\\
                       Emits 270 grams/kwh $\rightarrow$ 2.62 kg/litre. \\
                       On an average, for 20,000 km of distance travelled a year,(needs $\approx$ 670 litres of Diesel) car running on Diesel emits 1.74 tonnes of carbon dioxide in a year. \\ \\
                    3. CNG car. \\
                       Goes average 25 km/kg. \\
                       Consumes 14.8 kwh/kg of energy.\\
                       Emits 197 grams/kwh $\rightarrow$ 2.9 $kg_{c}$/kg. \\
                       On an average, for 20,000 km of distance travelled a year,(needs $\approx$ 800 kg of CNG) car running on CNG emits 2.34 tonnes of carbon dioxide in a year.\\ \\
                    4. Electric car. \\
                       Goes average 25 km/kg. \\
                       Consumes 23kwh/100km equivalent to 8.8kwh/L of energy.\\
                       Emits 527 grams/kwh $\rightarrow$ 4.63 $kg_{c}$/litre. \\
                       On an average, for 20,000 km of distance travelled a year,(needs $\approx$ 4600 kwh of electricity) car running on CNG emits 2.42 tonnes of carbon dioxide in a year.\\ \\
\end{minipage}} \\ \\
                  
\begin{tabular}{ |p{2.4cm}||p{2.8cm}|p{2.5cm}|p{2.6cm}|p{2.5cm}|p{2.5cm}| }
\hline
\multicolumn{6}{|c|}{$CO_{2}$ emission} \\
\hline
Car type & Energy Density ($kwh/L$) & Car Mileage ($km/L$) & C$O_{2}$emission  ($kg_{CO_{2}}$)/kwh & $CO_{2}$emission per litre & $CO_{2}$emission tC$O_{2}$/year\\
\hline
Gasoline car  & 9.5 & 20 & 0.24 & 2.28 & 2.28 \\
Diesel car  & 9.7 & 30 & 0.27 & 2.67 & 1.74 \\
CNG car    &  14.8 kwh/kg & 25 km/kg & 0.197 & 2.9 $kg_{c}$/kg &  2.34  \\
%Electric car  &  23km/100km & 4.3km/kwh & 0.527 & 12.1/100 km & 2.42\\
Electric car  &  8.8 & 38.46 & 0.527 & 4.6 & 2.42\\
\hline
\end{tabular} \\ \\
\textbf{} \\ \\ \\ \\
%%=================================================================== 


%%===================================================================                  
\question
\label{Q25:Grid emission factor of Singapore}
What is the grid emission factor (GEF) of Singapore? \\
\textbf{Step 1}. Required information \\ \\
                1. GEF measures average CO 2 emissions emitted per unit net electricity generated. It is calculated using the Average Operating Margin (OM) method. This is the generation-weighted average C$O_{2}$ emission per unit net electricity generation of all generating power plants serving the electricity grid.  \\ \\
                2. There are few ways to estimate the GEF of a nation based on the data available. Here a method is used based on the available data of total fuel consumption by Singapore in a year and electricity generation of the system. \\ The simple operating margin emission factor is calculated based on the net electricity supplied to the gridby all power plants serving the system and based on the fuel types and total fuel consumption of the project electricity system as follows:\\ \\
                
      \textbf{ $GEF_{OMsimple,y}$ = $\frac{\sum_{i}(FC_{i,y} \times NCV_{i,y} \times EF_{co_{2},i,y})}{EG_{y}}$} \\ \\
        where, \\
        1. $GEF_{OMsimple,y}$ = Simple operating margin $CO_{2}$ emission factor in year y ($tonne_{CO_{2}}$/MWh). \\
        
        2. $FC_{i,y}$  = Amount of fossil fuel type i consumed in the project electricity system in year y (mass or volume unit). \\
        
        3. $NCV_{i,y}$ = Net calorific value (energy content) of fossil fuel type i in year y (GJ/mass or GJ/volume unit).  \\
        
        4. $EF_{co_{2},i,y}$ = $CO_{2}$ emission factor of fossil fuel type i in year y ($tonne_{CO_{2}}$/GJ).\\
        
        5. $EG_{y}$ = Net electricity generated and delivered to the grid by all power sources serving the system (MWh). \\
        
        6. i = All fossil fuel types combusted in power sources in the project electricity system in year y. \\
        
        7. y = The relevant year as per the data vintage chosen. \\ 
        
        3. \textbf{Net Electricity generated and delivered to the grid = 50000 GWh = 50 $\times$ 10$^{9}$kwh.} \\ \\
            
                               
\begin{tabular}{ |p{3.4cm}||p{2.5cm}|p{2.5cm}|p{3.3cm}|p{3.3cm}| }
\hline
\multicolumn{5}{|c|}{$FC_{i,y}$} \\
\hline
Fuel type & MPPs & APPs & Total power     produced & Total fuel input used\\
\hline
Petroleum products  & 1.7 $\times$ 10$^{8}$ kwh & 22 $\times$ 10$^{8}$ kwh & 23 $\times$ 10$^{8}$ kwh & 1.9 $\times$ 10$^{8}$kg\\
Natural gas  & 9.4 $\times$ 10$^{10}$ kwh & 1.15$\times$10$^{10}$kwh & 10.55 $\times$ 10$^{10}$ kwh & 6.9 $\times$ 10$^{9}$kg\\
Coal and Peat  & 2.9 $\times$ 10$^{9}$ kwh & $\_$ & 2.9 $\times$ 10$^{9}$ kwh & 439 $\times$ 10$^{6}$kg\\
Others  & 7.9 $\times$ 10$^{9}$ kwh & 2.31$\times$10$^{8}$ kwh & 81.31 $\times$ 10$^{8}$ kwh & 774.3 $\times$ 10$^{6}$kg\\
\hline
\end{tabular} \\ 

where, MPPs = Major power producers. \\
       APPs = Auto power producers. \\

\begin{tabular}{ |p{4cm}||p{3cm}|p{3cm}| }
\hline
\multicolumn{3}{|c|}{$NCV_{i,y}$} \\
\hline
Fuel type & MJ/kg & kwh/kg \\
\hline
Petroleum products  & 44 & 12.5 \\
Coal  & 24 & 6.6 \\
Natural gas  & 55 & 15.27 \\
Others(Considering Biodiesel)  & 38 & 10.5 \\
\hline
\end{tabular} \\ \\

\begin{tabular}{ |p{4cm}||p{3cm}| }
\hline
\multicolumn{2}{|c|}{$EF_{co_{2},i,y}$} \\
\hline
Fuel type & $kg_{CO_{2}}$/kwh \\
\hline
Petroleum products  & 0.24 \\
Coal  & 0.38 \\
Natural gas  & 0.23 \\
\hline
\end{tabular}\\ \\
                
\fbox{\begin{minipage}{42em}
\textbf{Step 2}. %Consumption of energy/day in Singapore = 450 $\times$ 10$^{12}$ joules. \\  
                1. Petroleum products = $FC_{i,y} \times NCV_{i,y} \times EF_{co_{2},i,y}$ \\
                                      = 1.98 $\times$ 10$^{8}$ kg $\times$ 12.5 kwh /kg $\times$ 0.24 $kg_{CO_{2}}$/kwh \\
                                      = 5.7 $\times$ 10$^{8}$ $kg_{CO_{2}}$ \\ \\
                                      
                2. Natural gas        = $FC_{i,y} \times NCV_{i,y} \times EF_{co_{2},i,y}$ \\
                                      = 6.9 $\times$ 10$^{9}$ kg $\times$ 15.27 kwh /kg $\times$ 0.23 $kg_{CO_{2}}$/kwh \\
                                      = 2.42 $\times$ 10$^{10}$ $kg_{CO_{2}}$ \\ \\
                                      
                3. Coal = $FC_{i,y} \times NCV_{i,y} \times EF_{co_{2},i,y}$ \\
                                      = 439 $\times$ 10$^{6}$ kg $\times$ 6.6 kwh /kg $\times$ 0.38 $kg_{CO_{2}}$/kwh \\
                                      = 1.1 $\times$ 10$^{9}$ $kg_{CO_{2}}$ \\ \\
                                      
                So, $GEF_{OMsimple,y}$ = $\frac{\sum_{i}(5.7 \times 10^{8} \times 2.42 \times 10^{10} \times 1.1 \times 10^{9})}{50 \times 10^{9} kwh}$  \\
                                       = $\frac{258 \times 10^{11}grams}{50 \times 10^{12} wh}$ \\
                                       = 0.516 grams/wh \\

\end{minipage}} \\ \\
                 \textbf{Singapore's grid emission factor is 0.516 grams/wh.} \\ \\ 

%%=================================================================== 


%%=================================================================== 
\question
\label{Q25:Solar power}
What is the contribution of solar power in the total energy generated in Singapore in a year? \\
\textbf{Answer}. \\
\textbf{Step 1}. Required information \\
                 1. Total Energy generated per year in Singapore = 50 Twh.\\
                    
                 %1. Total Energy consumption per year in Singapore $\approx$ 48 Twh = 1.7 $\times$ 10$^{17}$ joules. $\Rightarrow$ $\approx$ 450 $\times$ 10$^{12}$ joules/day.\\ \\
                 2. There are total 900 solar PV installations across the island. This system gives 45.8 $MW_{ac}$ of power in 1 year.\\\\
                     45.8 MW = 45.8 $\times$ 10$^{6}$ $\times$ 3600 watt-hour.\\ 
                             = 1.65 $\times$ 10$^{11}$ watt-hour = 165 Gwh.\\ 
                    
\fbox{\begin{minipage}{42em}
\textbf{Step 2}. 1. Total power generated in a year = 50,000 Gwh. \\ \\
                    Power generated by solar system = 165 GWh. \\ \\
                    $\Rightarrow$ This is about 0.33\% of the total energy generated.\\
                   % Total power consumed in a year = 48,000 Gwh.\\
\end{minipage}} \\ \\
                   So, \textbf{Solar energy contribute only 0.33\% to the total energy generated in a year.}\\                    
                 
%%=================================================================== 


%%===================================================================                  
\question
\label{Q25:Solar Panel Singapore}
How much area of Singapore needs to be covered with solar panels in order to meet up the daily and yearly energy requirement of Singapore? \\
\textbf{Answer}. \\
\textbf{Step 1}. Required information \\
                 1.The amount of electricity a solar panel produces depends on three main things: 1. The size of the panel. \\
        2. The conversion efficiency of the solar cells.\\
        3. The amount of sunlight the panel receives.\\
                 2. Singapore's annual solar insolation = 1663 kwh/m$^{2}$ which is equivalent to receiving 4.55 peak Sun hours/ day.\\ (% Singapore is regarded as a favourable site for solar installations.) 
                 3. Peak Sun hours are number of hours per day when solar irradiance = 1000 w/m$^{2}$.\\
                 %4. Efficiency of a Crystalline Silicon solar panel = 15-20\% (about 150-200 w/m$^{2}$).\\
                 4. Dimensions of Solar panel = 1.65 m $\times$ 1.01 m $\times$ 0.046 m.\\
                    Area of a solar panel $\approx$ 1.65 m$^{2}$. \\
                    
                 5. Typical peak power output of a modern solar panel = 250 to 270 watts (or 250 to 270 Wp) in controlled conditions. However, when these panels are integrated into a system, the aggregate efficiency is low. Average peak power output of a system is 200 Wp/m$^{2}$. \\ \\
                 6. Efficiency of a solar panel from above information is $\frac{1000 w/m^{2}}{250 w/m^{2}}$ $\times$ 100 = 25\%. \\ \\
                 
                 7. Total Energy generated per year in Singapore = 50 Twh = 1.8 $\times$ 10$^{17}$ joules/year $\rightarrow$ $\approx$ 500 $\times$ 10$^{12}$ joules/day.
                                 
\fbox{\begin{minipage}{42em}
\textbf{Step 2}. 1. A 250 watt solar panel will produce 250 wh of energy in 1 hour. \\
                    $\Rightarrow$ For 5 hours of Sunlight, it will produce 1.25 kwh/day.\\ \\
                    
                 2. If 1.25 kwh/1.65 m$^{2}$ then $\Rightarrow$ 0.76 kwh/m$^{2}$ = 2.7 $\times$ 10$^{6}$ joules/m$^{2}$.\\ \\
                 
                 3. Singapore's energy consumption per day is 450 $\times$ 10$^{12}$ joules.\\
                    If 2.7 $\times$ 10$^{6}$ joules in 1 m$^{2}$, \\
                    then 450 $\times$ 10$^{12}$ joules $\rightarrow$ ? m$^{2}$\\ \\
                    $\Rightarrow$ $\frac{450 \times 10^{12} joules \times 1 m^{2}}{2.7 \times 10^{6} joules}$ $\approx$ 166 $\times$ 10$^{6}$ m$^{2}$ = 166 km$^{2}$. \\
                    This is about 23\% of the total area of Singapore. \\ \\
                    
                 However, as the efficiency of the solar systems is $\approx$ 5-10\% lower as compared to that of individual solar panels, the total area needed to install the solar systems to meet up the daily energy consumption of the Singapore will be 5-10\% more than the above calculated area. \\ \\
                 
                  \textbf{So, to meet up the daily and yearly energy requirement of Singapore, atleast 20-30\% of the Singapore's total area should be covered with Solar panels.}\\ \\
                   
                 5. Number of Solar panels required to cover 166 km$^{2}$ of area.\\
                    If 1.65 m$^{2}$ $\rightarrow$ 1 solar panel. \\
                    then 166 $\times$ 10$^{6}$ m$^{2}$ $\rightarrow$ ? solar panels. \\ \\
                    $\Rightarrow$ $\frac{166 \times 10^{6}}{1.65}$ = 100 million solar panels. \\ \\
           \textbf{So, approximately 100 million solar panels are required to cover the (166 km$^{2}$) of area to generate energy needed for Singapore's daily energy consumption.} \\ \\
           
                 6. How many number of solar panels are required to cover the entire area of Singapore? How much energy this system can produce? \\
                   If 1.65 m$^{2}$ $\rightarrow$ 1 solar panel. \\
                   then 719 $\times$ 10$^{6}$ m$^{2}$ $\rightarrow$ ? solar panels. \\ \\
                   $\Rightarrow$ $\frac{719 \times 10^{6}}{1.65}$ $\approx$ 440 million solar panels. \\ \\
                   If 0.76 kwh of energy is generated from 1 m$^{2}$  \\
                   then  how much energy is generated in 719 $\times$ 10$^{6}$ km$^{2}$? \\
                   $\Rightarrow$ 719 $\times$ 10$^{6}$ $\times$ 0.76 kwh = 546 $\times$ 10$^{6}$ kwh = 546 Gwh/day = $\approx$ 2000 $\times$ 10$^{12}$ joules/day  \\ \\
                   
                   \textbf{So, if the entire area of Singapore is to be covered with solar panels, we need 440 million solar panels and this system will produce 546 Gwh of energy per day. This is 4 times the energy produced per day in Singapore by non solar systems.}
                   
\end{minipage}} \\ \\                   
                   
%%=================================================================== 


%%===================================================================  
\question
\label{Q25:Solar power}
%What is the contribution of solar power in the total energy consumption in Singapore in a year? \\
%\textbf{Answer}. \\
%\textbf{Step 1}. Required information \\
%                 1. Total Energy consumption per year in Singapore $\approx$ 48 Twh \\
%                    Gasoline = 9.5 kwh/litre \\
%                    Diesel = 9.7 kwh/litre \\
%                    CNG = 53.6 MJ/kg = 14.8 kwh/kg\\ %(or 7.5 kwh/litre) \\
%                    %Biodiesel = 9.2 kwh/litre \\
%                    Electric car = 23 kwh/100km = 8.8 kwh/litre\\
%\fbox{\begin{minipage}{42em}
%\textbf{Step 2}

%\end{minipage}} \\ \\
%%=================================================================== 


%%===================================================================  
\question
\label{Q25:Solar power}
What is the contribution of solar power in the total energy consumption in Singapore in a year? \\
%\textbf{Answer}. \\
%\textbf{Step 1}. Required information \\
%                 1. Total Energy consumption per year in Singapore $\approx$ 48 Twh \\
%                    Gasoline = 9.5 kwh/litre \\
%                    Diesel = 9.7 kwh/litre \\
%                    CNG = 53.6 MJ/kg = 14.8 kwh/kg\\ %(or 7.5 kwh/litre) \\
%                    %Biodiesel = 9.2 kwh/litre \\
%                    Electric car = 23 kwh/100km = 8.8 kwh/litre\\
%\fbox{\begin{minipage}{42em}
%\textbf{Step 2}
%
%\end{minipage}} \\ \\
%%=================================================================== 


%%===================================================================  








%\question
%\label{Q16:Number of letters}
%Has humankind printed 1 mole of letters till now? If not then how much more time is required to print 1 mole of letters?
%\textbf{Step 1}. Required information \\
%                1. Dimensions of AA (Alkaline) batteries : \\
%                   Length = 5 cm \\ Diameter = 1.5cm \\ weight = 30 grams \\
%                2. Maximum energy at nominal voltage and 50 mA drain = 3 watt-hour = 10.8 $\times$ 10$^{3}$ joules. \\ \\
%                
%\fbox{\begin{minipage}{42em}
%\textbf{Step 2}. 1. Consumption of energy/day in Singapore = 450 $\times$ 10$^{12}$ joules. \\  \\
%\end{minipage}} \\ \\
%%=================================================================== 
%
%
%%=================================================================== 
%\question
%\label{Q17:Memory units}
%How many memory units does a human being has in his life?
%=================================================================== 


%=================================================================== 
%\question
%\label{Q4}



\end{questions}
%=================================================================== 
%=================================================================== 
 
\end{document} 