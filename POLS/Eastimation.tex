\documentclass[11pt]{exam}
\RequirePackage{amssymb, amsfonts, amsmath, latexsym, verbatim, xspace, setspace}


% By default LaTeX uses large margins.  This doesn't work well on exams; problems
% end up in the "middle" of the page, reducing the amount of space for students
% to work on them.
\usepackage[margin=1in]{geometry}

% For an exam, single spacing is most appropriate
\singlespacing
%===========================================================
\begin{document}
%\noindent
\title{\textbf{Estimation : Interesting questions and Back of the envelope calculations}}
\maketitle
%===========================================================
\subsection{Avogadro Number: A Mole.}
\begin{questions}

\question
\label{Q:Al Foil}
How long one has to roll a Aluminium foil to get 1 mole of Aluminium atoms?\\
\textbf{Answer }: 
\\ 
\textbf{Step 1}. Required information\\ 
                  Atomic mass of Aluminium = 27. This means 1 mole of    Aluminium(Al) atoms weigh 27 grams.\\
                  Density of Al = 2.7 grams/cubic centimetre.\\
                  Dimensions of household Al foil:\\ Length = 25        meter(2500cm), Width = 30cm, and Thickness = 14 microns(0.0014cm).\\ \\
\fbox{\begin{minipage}{28em}                  
        \textbf{Step 2}. Volume of Al foil for above dimensions.\\
        Volume = Width x Length x Thickness\\
               = 30 x 0.014 x 2500 cubic centimetre\\
               = 105 cubic centimetre\\ \\
        \textbf{Step 3}. Mass of the Al foil = Volume x Density\\
                                    = 105 x 2.7\\
                                    = 283.5 grams\\ \\
        \textbf{Step 4}. If 27 grams = 1 mole then 283.5 grams = ?\\
                so, 283.5/27 = 10.5 moles\\ \\
        \textbf{Step 5}. If 283.5 grams = 2500 cm then 27 grams = ?\\
                So, (27 x 2500)/283.5 = 238.09 cm \approx 2.4 metre\\
\end{minipage}} \\ \\                
        So, To get 1 mole of Al atoms we have to roll 2.4 meters of Al foil of given dimensions.

                                   
        
               
        
        

\question
\label{Q:Dollar bills}
 How much each of 2 SGD, 10 USD, and Rs.50 bills weigh? How many atoms are there in each of the bills (What fraction of a mole?)?
 
\question
\label{Hair}
How many hair are there on the scalp of average adult human? How big a animal has to be to have 1 mole of hair on his body?

\question
\label{Q:A4 paper}
How many sheets of A4 size paper collectively has 1 mole of carbon atoms?

\question
\label{Q:Styrofoam}
How big is 1 cubic volume of 1 mole of Styrofoam?

\question
\label{Q:Diamond}
How big is a diamond of 1 mole of carbon?

\question
\label{Q: A dot of ink}
How many moles of molecules are there in a dot of white board marker ink?

\question
\label{Q:Phone battery}
How many moles of electrons does a fully charged phone battery stores? 

\question
\label{Q:Shade Calories}
How many litres of iced water one has to drink to reduce 500 calories a day?

\question
\label{Q:Lightest morning}
Why we weigh lightest in the morning after the sleep? Where does this lost mass go?

\question
\label{Q:ATPs}
How many moles of ATP molecules does each cell in a human body synthesize? How many ATP molecules are required to run for 1 hour? And how many ATP molecules are required to run at a speed of 60Km/hr?

\question
\label{Q:Moles of water molecules}
How many moles of water molecules does a human consume in a day? And in his life span?

\question
\label{Q:Leaves on trees}
How many leaves per tree are there? Which type of trees will have dense leaves- trees with smaller leaves or trees with bigger leaves?

\question
\label{Q:Number of letters}
Has humankind printed 1 mole of letters till now? If not then how much more time is required to print 1 mole of letters?

\question
\label{Q:E.coli}
For a bacteria E. coli, how big a ball of 1 mole of E. coli will be? 

\question
\label{Q:E.coli and DNA basepairs}
How big a packed ball of E. coli will be, so that there are 1 mole of DNA base pairs?

\question
\label{Q:E.coli and water}
How big a ball of E. coli will be to have 1 mole of water?

\question
\label{Q:Memory units}
How many memory units does a human being has in his life?

%\question
%\label{Q4}



\end{questions} 
\end{document} 