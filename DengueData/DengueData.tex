%===================================================================
% Latex file: 11.dec.2016,Exam style:
%===================================================================
%===================================================================
% By default LaTeX uses large margins.  This doesn't work well on exams; problems
% end up in the "middle" of the page, reducing the amount of space for students
% to work on them.
% For an exam, single spacing is most appropriate
%===================================================================

\documentclass[11pt]{exam}
\RequirePackage{amssymb, amsfonts, amsmath, latexsym, verbatim, xspace, setspace,mathpazo}
\usepackage[margin=1in]{geometry}
\usepackage{textcomp}
\usepackage{graphicx}
\usepackage{float}
\usepackage{subfig}
\usepackage{hyperref}
\usepackage[utf8]{inputenc}
\usepackage[english]{babel}
\hypersetup{
%linkcolor=blue,
%citecolor=blue
colorlinks=true,
linkcolor=cyan,
filecolor=magenta,      
urlcolor=blue,
}
\urlstyle{same}

\graphicspath{ {/home/asawari/Desktop/DengueData/} }
\singlespacing

%===================================================================

\begin{document}
\title{\textbf{Literature Survey: Dengue conditions in Singapore}}
\maketitle

%===================================================================
%\subsection{Avogadro Number: A Mole.}
%===================================================================
\textbf{Index : List of questions.} \\ \\
Introduction: \\

1. Which animal causes maximum number of human deaths in a year? \\ \\
2. What is Dengue? How lethal is it? \\ \\
3. What causes Dengue?  \\ \\
4. How do mosquitoes carry a dengue virus? \\ \\
5. How do mosquitoes transmit dengue to a person? \\ \\%To how many people one dengue mosquito can infect with dengue virus? (directly (one to one) and indirectly (one to many)). How a dengue virus spread?\\ \\
6. What is the life cycle of a mosquito?  \\ \\
7. How fast the person gets infected with dengue after mosquito bite? What is the duration of mosquito bite to positive dengue condition? What are the intermediate stages of mosquito bite and positive dengue conditions?\\ \\
8. Which conditions are favourable for dengue mosquito breeding?\\ %and for non dengue mosquito breeding? \\
9. Which is the peak period and/or season of the year for dengue mosquito breeding and subsequent dengue hotspot/cluster emergence? \\ \\% What is dengue hotspot or dengue cluster?\\ \\
10. Which man-made locations are possible mosquito breeding sites? (E.g. Construction site, Sewage system, food hawker centres etc.)\\ \\
11. Is there any relation between average rainfall and number of dengue hotspot emergence? \\ \\
%Are there any records of rainfall and number of dengue cases registered per year? %(What is the ratio of mosquito breeding per mm of rainfall in an area (e.g. District) in Singapore?).\\ \\
12. Is it possible to make dengue mosquitoes harmless at any stage of their life cycle before they spread virus? If yes, what are the possible ways?\\ \\
13. Is it possible to locate possible dengue clusters well before in time (mosquito breeding) and eradicate those locations? What information one needs to know to locate the possible dengue cluster?\\ \\
14. Which conditions are unfavourable for dengue mosquito breeding? \\ \\
15. Is it possible to convert a favourable breeding spot to unfavourable spot to stop the possible mosquito breeding? \\ \\
16. What if all the mosquitoes in the world are extinguished? Will there be a missing link in the food chain? What would be the possible implications of this situation? \\ \\ \\

\textbf{Dengue numbers in Singapore.} \\ \\
17. Which are the dengue hotspots in Singapore? \\ \\
18. What is the difference in hygiene conditions in dengue hotspot areas and non hotspot 
    areas? \\ \\
19. How many positive dengue cases are recorded in each district of Singapore? \\ \\
20. How many deaths are recorded due to dengue in each year? What is the death rate due to dengue? \\ \\
21. What is the duration of positive dengue condition and complete recovery from dengue? What are further precautions to be taken in order prevent getting infected with dengue again? \\ \\
22. What treatments are used to treat a dengue patient? What is the cost of treatment and 
    medication? \\ \\ \\ \\ \\ \\ \\ \\ \\ \\ \\ \\ \\ \\ \\ \\ \\ \\ \\ \\ \\ \\ 
%23.

%===================================================================
\begin{questions}
%===================================================================
\question
\label{Q1:Most lethal animal}

\textbf{Which animal causes maximum number of human deaths in a year?}  \\
\textbf{Answer}: \textbf{Ref:}[1].

\begin{figure}[H]
  \centering
   \includegraphics[width=0.8\textwidth,natwidth=510,natheight=542]{WDA.png}
  \caption{World's deadliest animals.}
   \label{fig: World's deadliest animals }
\end{figure}
According to WHO report, every year more than one billion people are infected with vector-borne diseases and more than one million people die from vector-borne diseases including malaria,dengue, yellow fever, lymphatic filariasis etc. \\
Vectors are any living organisms that transmits infectious diseases between people or from animal to people.\\ %(Vector-borne diseases are the illnesses caused by parasites and pathogens in human populations.) \\
Many of these vectors are blood sucking insects that ingest disease-causing micro-organisms from the infected(virus carrying) host during a blood meal and later inject it into new host during their next blood meal.     \\
Mosquitoes are best know disease vectors. There are around 3500 species of mosquitoes. Some of these species have ability to carry many different diseases.\\
Among these species, \textit{Aedes Aegypti (Ae. aegypti)} is a primary vector that spreads dengue, Zika, chikungunya, and yellow fever.[2]\\ \\
\textbf{Mosquitoes are effective transmitters:} Mosquitoes feed on blood. When feeding they pierce the skin like a needle and inject a saliva into a person's skin. This transmits the disease causing agents into the site. As mosquitoes fly, they can spread diseases more quickly than any contagious disease like Ebola.[3] \\ \\
One sixth of the illness and disability suffered worldwide is due to vector-borne diseases with more than half the world's population currently estimated to be at the risk of these diseases.[2]  
Nearly 700 million people get mosquito-borne illness each year and more than one million of them result in death.[4] \\ 
\textbf{Summary:} \\
\fbox{\begin{minipage}{42em}  
1.  Mosquitoe being an effective transmitter of vector-borne diseases, is the deadliest insect causing around 1 million deaths per year by transmitting diseases like dengue, Zika, yellow fever, chikungunya etc.\\
2. There are around 3500 species of mosquitoes. \\
3. Nearly 700 million people get mosquito-borne illness each year.
\end{minipage}}\\ \\
%===================================================================

%=================================================================== 
\question
\label{Q2.Lethal dengue} 
\textbf{What is dengue? How lethal is it?}\\
\textbf{Answer}:\\
Dengue is a mosquito-borne viral infection that affects infants, young children and adults. This illness causes flue-like illness and occasionally develops into potentially lethal complication called \textit{severe dengue}[5].There is no specific treatment for dengue fever but early clinical diagnosis and careful clinical management by experienced physicians and nurses often save lives. \\
%There are over 2.5 billion people worldwide at risk of dengue infection and \textbf{20 million cases a year} are reported in more than 100 countries.[6]\\ \\ In 1995, the worst dengue epidemic in Latin America and the Caribbean for 15 years struck at least 14 countries, causing more than 200 000 cases of dengue fever and almost 6000 cases of the more serious dengue haemorrhagic fever.[6] \\ \\
The incidence of dengue has grown dramatically around the world in recent decades. The actual numbers of dengue cases are under reported and many cases are misclassified. One recent estimate indicates \textbf{390 million dengue infections per year} (95\% credible interval 284–528 million), of which \textbf{96 million (67–136 million) manifest clinically} (with any severity of disease). \\ Another study, of the prevalence of dengue, estimates that 3.9 billion people, in 128 countries, are at risk of infection with dengue viruses. \\ \\
Member States in 3 WHO regions regularly report the annual number of cases. The number of cases reported increased from 2.2 million in 2010 to 3.2 million in 2015. Although the full global burden of the disease is uncertain, the initiation of activities to record all dengue cases partly explains the sharp increase in the number of cases reported in recent years.[5]\\ 
\textbf{Summary:} \\
\fbox{\begin{minipage}{42em}  
1. Dengue is one of the most lethal disease.\\
2. Around 2.5 to 3.9 billion people worldwide are at risk of dengue infection. \\
3. 390 million dengue infections per year.\\
4. Increased number of 3.2 billion dengue cases reported in 2015 as compared to 2.2 million in 2010.
\end{minipage}}\\ \\
%===================================================================

%=================================================================== 
\question
\label{3. Dengue cause }
\textbf{What causes Dengue?}\\
\textbf{}
\textit{Aedes aegypti} mosquito is the primary vector of the dengue virus. The virus is transmitted to humans through the bites of female mosquitoes. \\
Vector-borne diseases like dengue are the illnesses caused by parasites and pathogens in human populations. Infected humans are the main carriers and multipliers of the virus, serving as a source of the virus for uninfected mosquitoes. Patients who are already infected with the dengue virus can transmit the infection (for 4–5 days; maximum 12) via Aedes mosquitoes once their first symptoms appear. \\ 

\begin{figure}[H]
  \centering
  \subfloat[Aedes Aegypti]{
  \includegraphics[width=0.4\textwidth,natwidth=50,natheight=80]{Aedes_aegypti.jpg}\label{fig:f1}}
  \hfill
  \subfloat[Aedes Albopictus]{  
  \includegraphics[width=0.4\textwidth,natwidth=50,natheight=80]{Aedes_Albopictus.jpg}\label{fig:f2}}
  \caption{Dengue vectors}
  \label{Dengue vectors }
\end{figure} 

\textit{Aedes albopictus}, a secondary dengue vector in Asia, has spread to North America and more than 25 countries in the European Region, largely due to the international trade in used tyres (a breeding habitat) and other goods (e.g. lucky bamboo). Ae. albopictus is highly adaptive and, therefore, can survive in cooler temperate regions of Europe. Its spread is due to its tolerance to temperatures below freezing, hibernation, and ability to shelter in microhabitats.\\
\textbf{Summary:} \\
\fbox{\begin{minipage}{42em}  
1. Two types of mosquitoes from same genus namely, Aedes aegypti and Aedes albopictus carry dengue virus from an infected person to another uninfected person.\\
2. These mosquitoes transmit dengue virus by injecting virus into the uninfected person's body through saliva during their blood meal. 
\end{minipage}}\\ \\
%===================================================================

%=================================================================== 
\question
\label{4. Mosquito: A dengue vector}
\textbf{How does a mosquito carry a dengue virus?} \\
%What chemicals in mosquito cause dengue virus to spread?}\\
\textbf{Answer:} \\
1. Typically, both male and female mosquitoes feed on nectar and plant juices. But in many species, to lay eggs, female mosquito needs to obtain protein from a blood meal. For this, in many species the mouthparts of the females are adapted for piercing the skin of animal hosts and sucking their blood. [7]\\
2.The mosquito becomes infective approximately seven days after it has bitten a person carrying the virus. This is the extrinsic incubation period, during which time the virus replicates in the mosquito and reaches the salivary glands.[8] \\

Mosquitoes carrying such arboviruses(any virus that is transmitted by an arthropod.) stay healthy because their immune systems recognizes the virions as foreign particles and "chop off" the virus's genetic coding, rendering it inert. Human infection with a mosquito-borne virus occurs when a female mosquito bites someone while its immune system is still in the process of destroying the virus's harmful coding. \\
It is not completely known how mosquitoes handle eukaryotic parasites to carry them without being harmed. Data has shown that the malaria parasite Plasmodium falciparum alters the mosquito vector's feeding behaviour by increasing frequency of biting in infected mosquitoes, thus increasing the chance of transmitting the parasite.[7]\\
\textbf{Summary:} \\
\fbox{\begin{minipage}{42em}  
1. Only female mosquitoes need protein in the blood for laying eggs. Hence, only female mosquitoes carry dengue virus from infected person during their blood meal.\\
2. These mosquitoes become infective approximately seven days after their infected blood meal. During this incubation period virus replicates in the mosquitoes and reaches the salivary glands.\\
3. Mosquitoes carrying these viruses are immune to those foreign particles.
\end{minipage}}\\ \\

%===================================================================

%=================================================================== 
\question
\label{5. Dengue transmission }
\textbf{How do mosquitoes transmit dengue to a person?}\\
\textbf{Answer:}\\
Once an infected mosquito has incubated the virus for 4–10 days, it can transmit the virus for
the rest of its life.[5] \\ 
Prior to and during blood feeding, blood-sucking mosquitoes inject saliva into the bodies of their source(s) of blood. This saliva serves as an anticoagulant %(; without it one might expect the female mosquito's proboscis to become clogged with blood clots.)
and the main route by which mosquito physiology offers passenger pathogens access to the hosts' interior. The salivary glands are a major target to most pathogens, whence they find their way into the host via the stream of saliva.[7] \\ 
%\textbf{To how many people one dengue mosquito can infect with dengue virus? (directly (one to one) and indirectly (one to many)). How a dengue virus spread?}\\ 

\textbf{Interesting facts about mosquito feed and hunt for blood hosts:} \\ 

1. All adult mosquitoes feed on the nectar or honey dew of plants to get sugar, and that provides enough nourishment for both males and females to live. %(Which means Male mosquitoes do not suck bloods, only female mosquitoes do for the need of reproduction!) \\ 
Both plant materials and blood are useful sources of energy in the form of sugars, and blood also supplies more concentrated nutrients, such as lipids, but the most important function of blood meals is to obtain proteins as materials for egg production.
2. A mosquito has a variety of ways of finding its prey, including chemical, visual, and heat sensors. \\
3. \textbf{Feeding preferences of mosquitoes:} Those with type O blood, heavy breathers, those with a lot of skin bacteria, people with a lot of body heat, and the pregnant.\\
4. When a female reproduces with such parasitic meals, this reproduction is termed as \textbf{anautogenous}('anautogeny' is the condition found in many insects, where a gravid female has to feed before laying eggs in order for the eggs to mature.), as occurs in mosquito species that serve as disease vectors, particularly Anopheles and \textbf{Aedes}.\\
5. \textbf{Hunting for host location:} female mosquitoes hunt their blood host by detecting organic substances such as carbon dioxide (CO2) and 1-octen-3-ol (octenol for short and also known as mushroom alcohol, is a chemical that attracts biting insects such as mosquitoes.) produced from the host, and through optical recognition. Mosquitoes prefer some people over others. The preferred victim's sweat simply smells better than others' because of the proportions of the carbon dioxide, octenol and other compounds that make up body odor. \\
Another compound identified in human blood that attracts mosquitoes is sulcatone or 6-methyl-5-hepten-2-one, especially for \textbf{Aedes aegypti} mosquitoes with the odor receptor gene Or4. \\
A large part of the mosquito’s sense of smell, or olfactory system, is devoted to sniffing out blood sources. Of 72 types of odor receptors on its antennae, at least 27 are tuned to detect chemicals found in perspiration. \\
In \textbf{Aedes}, the search for a host takes place in two phases. First, the mosquito exhibits a nonspecific searching behavior until the perception of host stimulants, then it follows a targeted approach.[7] \\
6. \textbf{How mosquitoes digest the nectar and blood?}: After they draw food in through their long proboscis, they store it in a sack-like crop connected to the fore-gut. The fore-gut is divided into two parts, a front section that digests sugars and a rear section that digests blood. The front mid-gut secretes enzymes which digest the nectar into a liquid and it passes from the crop into the front mid-gut.\\ When the female mosquito feeds on blood, the blood goes directly to the rear midgut, which can digest the protein. The rear midgut can also expand as the mosquito draws in the blood, so she can hold a full meal at once. \\

\begin{figure}[H]
  \centering
  \subfloat[Mosquito before feeding.]    {\includegraphics[width=0.39\textwidth,natwidth=50,natheight=65]{mosquito_empty.jpg}\label{fig:f1}
}    \hfill
  \subfloat[Mosquito after feeding.] {\includegraphics[width=0.4\textwidth,natwidth=50,natheight=70]{mosquito_full.jpg}\label{fig:f2}}
  \caption{Mosquito before and after feeding.}
  \label{Dengue vectors }
\end{figure} 

\textbf{Summary:} \\
\fbox{\begin{minipage}{42em}  
1. Once the mosquito has incubated the dengue virus, it can transmit the virus for rest of its life.\\
2. During the blood meal on uninfected person's body, mosquito carrying dengue virus injects saliva into the person's body. This saliva serves the main route by which pathogens find their way into the host's interior. \\
3. Feeding preferences: Those with type O blood, heavy breathers, those with a lot of skin bacteria, people with a lot of body heat, and the pregnant.\\
4. Hunting for host location: By detecting organic substances like carbon dioxide, octenol, and  sulcatone produced from the host and through optical recognition.
\end{minipage}}\\ \\
%===================================================================

%=================================================================== 
\question
\label{6. Life cycle of Ae.aegypti}
\textbf{What is the life cycle of Aedes Aegypti?}\\
\textbf{Answer:} \\
Aedes Aegypti is a so-called holometabolous insect. This means that the insects goes through a complete metamorphosis with an egg, larvae, pupae, and adult stage. The life cycle of Aedes aegypti can be completed within one-and-a-half to three weeks. \\
\begin{figure}[H]
  \centering
  \includegraphics[width=0.6\textwidth,natwidth=80,natheight=60]{MosquitoLifeCycle.jpg}\label{fig:f1}
   %\hfill
%  \subfloat[Aedes Aegypti life cycle.[11]] {\includegraphics[width=0.4\textwidth,natwidth=50,natheight=70]{AedesAegyptiLifeCycle.jpg}\label{fig:f2}}
  \caption{Mosquito life cycle.[10]}
%  \label{Dengue vectors }
\end{figure} 
1. \textbf{Egg:} After taking a blood meal, female Aedes aegypti mosquitos produce on average 100 to 200 eggs per batch. The females can produce up to five batches of eggs during a lifetime. \\ %(The number of eggs is dependent on the size of the bloodmeal.)
Eggs are laid on damp surfaces in areas likely to temporarily flood, such as tree holes and man-made containers and a lot more places where rain-water collects or is stored.\\%(like barrels, drums, jars, pots, buckets, flower vases, plant saucers, tanks, discarded bottles, tins, tyres, water cooler, etc. ) \\
The female Aedes aegypti lays her eggs separately unlike most species. Not all eggs are laid at once, but they can be spread out over hours or days, depending on the availability of suitable substrates. Eggs will most often be placed at varying distances above the water line. The female mosquito will not lay the entire clutch at a single site, but rather spread out the eggs over several sites.\\

The eggs of Aedes aegypti are smooth, long, ovoid shaped, and roughly one millimeter long. When first laid, eggs appear white but within minutes turn a shiny black. In warm climates eggs may develop in as little as two days, whereas in cooler temperate climates, development can take up to a week. Laid eggs can survive for very long periods in a dry state, often for more than a year. However, they hatch immediately once submerged in water. This makes the control of the dengue virus mosquito very difficult.\\ \\

2. \textbf{Larvae:} After hatching of the eggs, the larvae feed on organic particulate matter in the water, such as algae and other microscopic organisms. Most of the larval stage is spent at the water's surface.\\ %(although they will swim to the bottom of the container if disturbed or when feeding.) 
Larvae are often found around the home in puddles, tires, or within any object holding water. Larval development is temperature dependent. Males develop faster than females, so males generally pupate earlier. If temperatures are cool, Aedes aegypti can remain in the larval stage for months so long as the water supply is sufficient. \\  The larvae pass through four instars, spending a short amount of time in the first three, and up to three days in the fourth instar. Fourth instar larvae are approximately eight millimeters long. 
Males develop faster than females, so males generally pupate earlier. If temperatures are cool, Aedes aegypti can remain in the larval stage for months so long as the water supply is sufficient.\\ 
\begin{figure}[H]
  \centering
   \subfloat[Ae.aegypti larvae stage] {\includegraphics[width=0.3\textwidth,natwidth=50,natheight=50]{AA_larvae.jpg}\label{fig:f1}}
    \hfill
    \subfloat[Ae.aegypti pupae stage.]    {\includegraphics[width=0.24\textwidth,natwidth=60,natheight=50]{AA_pupa.jpg}\label{fig:f2}
}  
  \caption{Aedes aegypti larvae and pupae stage.}
  \label{Ae.Aegypti life stages.}    
\end{figure} 

3.\textbf{Pupae:} After the fourth instar, the larvae enters the pupal stage. Mosquito pupae are mobile and respond to stimuli. Pupae do not feed and take approximately two days to develop. Adults emerge by ingesting air to expand the abdomen thus splitting open the pupal case and emerge head first.[11]\\

\textbf{Summary:} \\
\fbox{\begin{minipage}{42em}  
1. Aedes Aegypti goes through a complete metamorphosis with an egg, larvae, pupae, and adult stage.\\
2. After a blood meal,female Ae. aegypti mosquitoes lay upto five batches of 100 to 200 eggs. The eggs are laid on damped surface areas over several sites. The eggs can lie dormant in dry conditions for up to about nine months, after which they can hatch if exposed to     favourable conditions, i.e. water and food.[10] \\
3. Larval stage is temperature dependant. It passes through 4 instars. Males develop faster than females. If temperatures are cool, it remains in the larval stage for months as long as the water supply is sufficient.\\
4. Pupae do not feed and take approximately two days to develop into adult mosquito. The adult life span can range from two weeks to a month depending on environmental conditions.
\end{minipage}}\\ \\
%===================================================================

%=================================================================== 
%\question
%\label{3. Dengue: cause and Transmission }





%===================================================================

%=================================================================== 
%\question
%\label{3. Dengue: cause and Transmission }






%===================================================================

%=================================================================== 
%\question
%\label{3. Dengue: cause and Transmission }





%===================================================================

%=================================================================== 
%\question
%\label{3. Dengue: cause and Transmission }









%===================================================================

%=================================================================== 
%\question
%\label{3. Dengue: cause and Transmission }







%===================================================================


\end{questions}
%=================================================================== 
\textbf{References:} \\ \\
%Most lethal animal
 1] \url{https://www.ksl.com/?sid=29721017&nid=711} \\
 2] \url{http://apps.who.int/iris/bitstream/10665/111008/1/WHO_DCO_WHD_2014.1_eng.pdf}\\
 3] \url{http://www.healthline.com/health-news/mosquitos-the-most-dangerous-animal-on-earth-021216#4} \\
 4] \url{https://en.wikipedia.org/wiki/Mosquito-borne_disease} \\
%Lethal dengue
 5] \url{http://www.who.int/mediacentre/factsheets/fs117/en/}\\
 6] \url{http://www.who.int/whr/1996/media_centre/executive_summary1/en/index9.html} \\
%Dengue cause and transmission
 7] \url{https://en.wikipedia.org/wiki/Mosquito#Feeding_by_adults}\\
 8] \url{http://www.dengue.gov.sg/subject.asp?id=12} \\
%Interesting facts about mosquito feed and hunt for blood host.
 9] \url{http://www.mosquitoreviews.com/mosquitoes-eat.html}\\
%Lifecycle of Ae. Aegypti
 10] \url{http://www.dengue.gov.sg/subject.asp?id=12}\\
 11] \url{http://www.denguevirusnet.com/life-cycle-of-aedes-aegypti.html}\\
\end{document} 