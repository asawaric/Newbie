%===================================================================
% Latex file: 11.dec.2016,Exam style:
%===================================================================
%===================================================================
% By default LaTeX uses large margins.  This doesn't work well on exams; problems
% end up in the "middle" of the page, reducing the amount of space for students
% to work on them.
% For an exam, single spacing is most appropriate
%===================================================================

\documentclass[11pt]{exam}
\RequirePackage{amssymb, amsfonts, amsmath, latexsym, verbatim, xspace, setspace,mathpazo}
\usepackage[margin=1in]{geometry}
\usepackage{textcomp}
\usepackage{graphicx}
\usepackage{float}
\usepackage{subfig}
\usepackage{hyperref}
\usepackage[utf8]{inputenc}
\usepackage[english]{babel}
\hypersetup{
%linkcolor=blue,
%citecolor=blue
colorlinks=true,
linkcolor=cyan,
filecolor=magenta,      
urlcolor=blue,
}
\urlstyle{same}

\graphicspath{ {/home/asawari/Desktop/DengueData/Images/} }
\singlespacing

%===================================================================
\begin{document}
\title{\textbf{Literature Survey: Dengue conditions in Singapore}}
\maketitle
%===================================================================
%\subsection{Avogadro Number: A Mole.}
%===================================================================
\textbf{Index : List of questions.} \\ \\
Introduction:\\

1. Which animal causes maximum number of human deaths in a year? \\ \\
2. What is Dengue? How lethal is it? \\ \\
3. What causes Dengue?  \\ \\
4. How do mosquitoes carry a dengue virus? \\ \\
5. How do mosquitoes transmit dengue to a person? \\ \\%To how many people one dengue mosquito can infect with dengue virus? (directly (one to one) and indirectly (one to many)). How a dengue virus spread?\\ \\
6. What is the life cycle of a mosquito?  \\ \\
7. What are the symptoms of dengue infection? What is the duration of mosquito bite to positive dengue condition? What are the intermediate stages of mosquito bite and positive dengue conditions?\\ \\
8. Which conditions are favourable for dengue mosquito breeding?\\ %and for non dengue mosquito breeding? \\
9. Which is the peak period and/or season of the year for dengue mosquito breeding and subsequent dengue hotspot/cluster emergence? \\ \\% What is dengue hotspot or dengue cluster?\\ \\
10. Which man-made locations are possible mosquito breeding sites? (E.g. Construction site, Sewage system, food hawker centres etc.)\\ \\
11. Is there any relation between average rainfall and number of dengue hotspot emergence? \\ \\
%Are there any records of rainfall and number of dengue cases registered per year? %(What is the ratio of mosquito breeding per mm of rainfall in an area (e.g. District) in Singapore?).\\ \\
12. Is it possible to make dengue mosquitoes harmless at any stage of their life cycle before they spread virus? If yes, what are the possible ways?\\ \\
13. Is it possible to locate possible dengue clusters well before in time (mosquito breeding) and eradicate those locations? What information one needs to know to locate the possible dengue cluster?\\ \\
14. Which conditions are unfavourable for dengue mosquito breeding? \\ \\
15. Is it possible to convert a favourable breeding spot to unfavourable spot to stop the possible mosquito breeding? \\ \\
16. What if all the mosquitoes in the world are extinguished? Will there be a missing link in the food chain? What would be the possible implications of this situation? \\ \\

\textbf{Dengue numbers in Singapore.} \\ \\
17. Which are the dengue hotspots in Singapore? \\ \\
18. What is the difference in hygiene conditions in dengue hotspot areas and non hotspot 
    areas? \\ \\
19. How many positive dengue cases are recorded in each district of Singapore? \\ \\
20. How many deaths are recorded due to dengue in each year? What is the death rate due to dengue? \\ \\
21. What is the duration of positive dengue condition and complete recovery from dengue? What are further precautions to be taken in order prevent getting infected with dengue again? \\ \\
22. What treatments are used to treat a dengue patient? What is the cost of treatment and 
    medication? \\ \\ \\ \\ \\ \\ \\ \\ \\ \\ \\ \\ \\ \\ \\ \\ \\ \\ \\ \\ \\ \\ 
%23.

%===================================================================
\begin{questions}
%===================================================================
\question
\label{Q1:Most lethal animal}
\textbf{Which animal causes maximum number of human deaths in a year?}  \\
\textbf{Answer}:
\begin{figure}[H]
  \centering
   \includegraphics[width=0.8\textwidth,natwidth=510,natheight=542]{WDA.png}
  \caption{World's deadliest animals.[1]}
   \label{fig: World's deadliest animals }
\end{figure}
According to WHO report, every year more than one billion people are infected with vector-borne diseases and more than one million people die from vector-borne diseases including malaria,dengue, yellow fever, lymphatic filariasis etc. \\
Vectors are any living organisms that transmits infectious diseases between people or from animal to people.\\ %(Vector-borne diseases are the illnesses caused by parasites and pathogens in human populations.) \\
Many of these vectors are blood sucking insects that ingest disease-causing micro-organisms from the infected(virus carrying) host during a blood meal and later inject it into new host during their next blood meal.     \\
Mosquitoes are best know disease vectors. There are around 3500 species of mosquitoes. Some of these species have ability to carry many different diseases.\\
Among these species, \textit{Aedes Aegypti (Ae. aegypti)} is a primary vector that spreads dengue, Zika, chikungunya, and yellow fever.[2]\\ \\
\textbf{Mosquitoes are effective transmitters:} Mosquitoes feed on blood. When feeding they pierce the skin like a needle and inject a saliva into a person's skin. This transmits the disease causing agents into the site. As mosquitoes fly, they can spread diseases more quickly than any contagious disease like Ebola.[3] \\ \\
One sixth of the illness and disability suffered worldwide is due to vector-borne diseases with more than half the world's population currently estimated to be at the risk of these diseases.[2]  
Nearly 700 million people get mosquito-borne illness each year and more than one million of them result in death.[4] \\ 

\textbf{Summary:} \\
\fbox{\begin{minipage}{42em}  
1.  Mosquito being an effective transmitter of vector-borne diseases, is the deadliest insect causing around 1 million deaths per year by transmitting diseases like dengue, Zika, yellow fever, chikungunya etc.\\
2. There are around 3500 species of mosquitoes. \\
3. Nearly 700 million people get mosquito-borne illness each year.
\end{minipage}}\\ \\
%===================================================================

%=================================================================== 
\question
\label{Q2.Lethal dengue} 
\textbf{What is dengue? How lethal is it?}\\
\textbf{Answer}:\\
Dengue is a mosquito-borne viral infection that affects infants, young children and adults. It is caused by four closely related viruses (DEN-1, DEN-2, DEN-3 or DEN-4).
This illness causes flue-like illness and occasionally develops into potentially lethal complication called \textit{severe dengue}[5].The two most severe forms of dengue
are Dengue Haemorraghic Fever (DHF) and Dengue Shock Syndrome (DSS). \\ There is no specific treatment for dengue fever but early clinical diagnosis and careful clinical management often save lives. \\
%There are over 2.5 billion people worldwide at risk of dengue infection and \textbf{20 million cases a year} are reported in more than 100 countries.[6]\\ \\ In 1995, the worst dengue epidemic in Latin America and the Caribbean for 15 years struck at least 14 countries, causing more than 200 000 cases of dengue fever and almost 6000 cases of the more serious dengue haemorrhagic fever.[6] \\ \\
The incidence of dengue has grown dramatically around the world in recent decades. The actual numbers of dengue cases are under reported and many cases are misclassified. One recent estimate indicates \textbf{390 million dengue infections per year} (95\% credible interval 284–528 million), of which \textbf{96 million (67–136 million) manifest clinically} (with any severity of disease). \\ Another study, of the prevalence of dengue, estimates that 3.9 billion people, in 128 countries, are at risk of infection with dengue viruses. \\ \\
Member States in 3 WHO regions regularly report the annual number of cases. The number of cases reported increased from 2.2 million in 2010 to 3.2 million in 2015. Although the full global burden of the disease is uncertain, the initiation of activities to record all dengue cases partly explains the sharp increase in the number of cases reported in recent years.[5]\\ 

\textbf{About Dengue Virus :} \\
Dengue virus (DENV) is a tiny structure that cause the most common arthropod-borne viral disease in man with 50–100 million infections per year. \\
It is a mosquito-borne single positive-stranded RNA virus of the family Flaviviridae; genus Flavivirus. Four serotypes of the virus have been found, all of which can cause the full spectrum of disease. All four DENV serotypes have emerged from sylvatic strains in the forests of South-East Asia. \\
DENV infection can be asymptomatic or a self-limited, acute febrile disease ranging in severity. [22]\\
\begin{figure}[H]
  \centering
  \subfloat[Dengue virus structure]{
  \includegraphics[width=0.4\textwidth,natwidth=50,natheight=80]{DengueVirus.jpg}\label{fig:f1}}%[23]
  \hfill
  \subfloat[Structural components in Cross-section view of DNV]{  
  \includegraphics[width=0.4\textwidth,natwidth=50,natheight=80]{DNVCrossSection.jpg}\label{fig:f2}}%[24]
  \hfill
  \subfloat[TEM Micrograph showing DNV virions cluster]{  
  \includegraphics[width=0.4\textwidth,natwidth=50,natheight=80]{DNV_TEM.jpg}\label{fig:f3}}%[24]
  \caption{Dengue virus}
  \label{Dengue vectors }
\end{figure} 

The dengue virus can only replicate inside a host organism. It is a roughly spherical structure composed of the viral genome and capsid proteins surrounded by an envelope and a shell of proteins. The envelope is a lipid bilayer that is taken from the host. Embedded in the viral envelope are E and M proteins that span through the lipid bilayer. These proteins form a protective outer layer that controls the entry of the virus into human cells. \\
After infecting a host cell, the dengue virus hijacks the host cell's machinery to replicate the viral RNA genome and viral proteins. After maturing, the newly synthesized dengue viruses are released and go on to infect other host cells.[23]\\

\textbf{Summary:} \\
\fbox{\begin{minipage}{42em}  
1. Dengue is one of the most lethal disease.\\
2. Around 2.5 to 3.9 billion people worldwide are at risk of dengue infection. \\
3. 390 million dengue infections per year.\\
4. Increased number of 3.2 billion dengue cases reported in 2015 as compared to 2.2 million in 2010.
\end{minipage}}\\ \\ \\
%===================================================================

%=================================================================== 
\question
\label{3. Dengue cause }
\textbf{What causes Dengue?}\\
\textbf{Answer:}
\textit{Aedes aegypti} mosquito is the primary vector of the dengue virus. The virus is transmitted to humans through the bites of female mosquitoes. \\
Vector-borne diseases like dengue are the illnesses caused by parasites and pathogens in human populations. Infected humans are the main carriers and multipliers of the virus, serving as a source of the virus for uninfected mosquitoes. Patients who are already infected with the dengue virus can transmit the infection (for 4–5 days; maximum 12) via Aedes mosquitoes once their first symptoms appear. \\ 
\begin{figure}[H]
  \centering
  \subfloat[Aedes Aegypti]{
  \includegraphics[width=0.4\textwidth,natwidth=50,natheight=80]{Aedes_aegypti.jpg}\label{fig:f1}}
  \hfill
  \subfloat[Aedes Albopictus]{  
  \includegraphics[width=0.4\textwidth,natwidth=50,natheight=80]{Aedes_Albopictus.jpg}\label{fig:f2}}
  \caption{Dengue vectors}
  \label{Dengue vectors }
\end{figure} 

\textit{Aedes albopictus} also called as \textit{Asian tiger mosquito}, is a secondary dengue vector in Asia, has spread to North America and more than 25 countries in the European Region, largely due to the international trade in used tyres (a breeding habitat) and other goods (e.g. lucky bamboo). Ae. albopictus is highly adaptive and, therefore, can survive in cooler temperate regions of Europe. Its spread is due to its tolerance to temperatures below freezing, hibernation, and ability to shelter in microhabitats.\\

\textbf{Summary:} \\
\fbox{\begin{minipage}{42em}  
1. Two types of mosquitoes from same genus namely, Aedes aegypti and Aedes albopictus carry dengue virus from an infected person to another uninfected person.\\
2. These mosquitoes transmit dengue virus by injecting virus into the uninfected person's body through saliva during their blood meal. 
\end{minipage}}\\ \\
%===================================================================

%=================================================================== 
\question
\label{4. Mosquito: A dengue vector}
\textbf{How does a mosquito carry a dengue virus?} \\
%What chemicals in mosquito cause dengue virus to spread?}\\
\textbf{Answer:} \\
1. Typically, both male and female mosquitoes feed on nectar and plant juices. But in many species, to lay eggs, female mosquito needs to obtain protein from a blood meal. For this, in many species the mouthparts of the females are adapted for piercing the skin of animal hosts and sucking their blood. [7]\\
2. The mosquito becomes infective approximately seven days after it has bitten a person carrying the virus. This is the extrinsic incubation period, during which time the virus replicates in the mosquito and reaches the salivary glands.[8] \\

Mosquitoes carrying such arboviruses(any virus that is transmitted by an arthropod.) stay healthy because their immune systems recognizes the virions as foreign particles and "chop off" the virus's genetic coding, rendering it inert. Human infection with a mosquito-borne virus occurs when a female mosquito bites someone while its immune system is still in the process of destroying the virus's harmful coding. \\
It is not completely known how mosquitoes handle eukaryotic parasites to carry them without being harmed. Data has shown that the malaria parasite Plasmodium falciparum alters the mosquito vector's feeding behaviour by increasing frequency of biting in infected mosquitoes, thus increasing the chance of transmitting the parasite.[7]\\

\textbf{Summary:} \\
\fbox{\begin{minipage}{42em}  
1. Only female mosquitoes need protein in the blood for laying eggs. Hence, only female mosquitoes carry dengue virus from infected person during their blood meal.\\
2. These mosquitoes become infective approximately seven days after their infected blood meal. During this incubation period virus replicates in the mosquitoes and reaches the salivary glands.\\
3. Mosquitoes carrying these viruses are immune to those foreign particles.
\end{minipage}}\\ \\

%===================================================================

%=================================================================== 
\question
\label{5. Dengue transmission }
\textbf{How do mosquitoes transmit dengue to a person?}\\
\textbf{Answer:}\\
Once an infected mosquito has incubated the virus for 4–10 days, it can transmit the virus for
the rest of its life.[5] \\ 
Prior to and during blood feeding, blood-sucking mosquitoes inject saliva into the bodies of their source(s) of blood. This saliva serves as an anticoagulant %(; without it one might expect the female mosquito's proboscis to become clogged with blood clots.)
and the main route by which mosquito physiology offers passenger pathogens access to the hosts' interior. The salivary glands are a major target to most pathogens, whence they find their way into the host via the stream of saliva.[7] \\ 
%\textbf{To how many people one dengue mosquito can infect with dengue virus? (directly (one to one) and indirectly (one to many)). How a dengue virus spread?}\\ 

\textbf{Interesting facts about mosquito feed and hunt for blood hosts:} \\ 

1. All adult mosquitoes feed on the nectar or honey dew of plants to get sugar, and that provides enough nourishment for both males and females to live. %(Which means Male mosquitoes do not suck bloods, only female mosquitoes do for the need of reproduction!) \\ 
Both plant materials and blood are useful sources of energy in the form of sugars, and blood also supplies more concentrated nutrients, such as lipids, but the most important function of blood meals is to obtain proteins as materials for egg production.
2. A mosquito has a variety of ways of finding its prey, including chemical, visual, and heat sensors. \\
3. \textbf{Feeding preferences of mosquitoes:} Those with type O blood, heavy breathers, those with a lot of skin bacteria, people with a lot of body heat, and the pregnant.\\
4. When a female reproduces with such parasitic meals, this reproduction is termed as \textbf{anautogenous}('anautogeny' is the condition found in many insects, where a gravid female has to feed before laying eggs in order for the eggs to mature.), as occurs in mosquito species that serve as disease vectors, particularly Anopheles and \textbf{Aedes}.\\
5. \textbf{Hunting for host location:} female mosquitoes hunt their blood host by detecting organic substances such as carbon dioxide (CO2) and 1-octen-3-ol (octenol for short and also known as mushroom alcohol, is a chemical that attracts biting insects such as mosquitoes.) produced from the host, and through optical recognition. Mosquitoes prefer some people over others. The preferred victim's sweat simply smells better than others' because of the proportions of the carbon dioxide, octenol and other compounds that make up body odor. \\
Another compound identified in human blood that attracts mosquitoes is sulcatone or 6-methyl-5-hepten-2-one, especially for \textbf{Aedes aegypti} mosquitoes with the odor receptor gene Or4. \\
A large part of the mosquito’s sense of smell, or olfactory system, is devoted to sniffing out blood sources. Of 72 types of odor receptors on its antennae, at least 27 are tuned to detect chemicals found in perspiration. \\
In \textbf{Aedes}, the search for a host takes place in two phases. First, the mosquito exhibits a nonspecific searching behavior until the perception of host stimulants, then it follows a targeted approach.[7] \\
6. Both dengue vectors Ae aegypti and Ae. Albopictus are daytime feeders: The peak biting periods are early in the morning and in the evening before dusk. \\
 % Most mosquito species are crepuscular (dawn or dusk) feeders. During the heat of the day, most mosquitoes rest in a cool place and wait for the evenings, although they may still bite if disturbed. Some species, such as the Asian tiger mosquito, are known to fly and feed during daytime. 
%6. \textbf{How mosquitoes digest the nectar and blood?}: After they draw food in through their long proboscis, they store it in a sack-like crop connected to the fore-gut. The fore-gut is divided into two parts, a front section that digests sugars and a rear section that digests blood. The front mid-gut secretes enzymes which digest the nectar into a liquid and it passes from the crop into the front mid-gut.\\ When the female mosquito feeds on blood, the blood goes directly to the rear midgut, which can digest the protein. The rear midgut can also expand as the mosquito draws in the blood, so she can hold a full meal at once. \\
\begin{figure}[H]
  \centering
  \subfloat[Mosquito before feeding.]    {\includegraphics[width=0.39\textwidth,natwidth=50,natheight=65]{mosquito_empty.jpg}\label{fig:f1}
}    \hfill
  \subfloat[Mosquito after feeding.] {\includegraphics[width=0.4\textwidth,natwidth=50,natheight=70]{mosquito_full.jpg}\label{fig:f2}}
  \caption{Mosquito before and after feeding.}
  \label{Dengue vectors }
\end{figure} 

\textbf{Summary:} \\
\fbox{\begin{minipage}{42em}  
1. Once the mosquito has incubated the dengue virus, it can transmit the virus for rest of its life.\\
2. During the blood meal on uninfected person's body, mosquito carrying dengue virus injects saliva into the person's body. This saliva serves the main route by which pathogens find their way into the host's interior. \\
3. Feeding preferences: Those with type O blood, heavy breathers, those with a lot of skin bacteria, people with a lot of body heat, and the pregnant.\\
4. Hunting for host location: By detecting organic substances like carbon dioxide, octenol, and  sulcatone produced from the host and through optical recognition.\\
5. Peak biting period of dengue vector mosquitoes is daylight; approximately two hours after sunrise and several hours before sunset.
\end{minipage}}\\ \\
%===================================================================

%=================================================================== 
\question
\label{6. Life cycle of Ae.aegypti}
\textbf{What is the life cycle of Aedes Aegypti?}\\
\textbf{Answer:} \\
Aedes Aegypti is a so-called holometabolous insect. This means that the insects goes through a complete metamorphosis with an egg, larvae, pupae, and adult stage. The life cycle of Aedes aegypti can be completed within one-and-a-half to three weeks. \\
\begin{figure}[H]
  \centering
  \includegraphics[width=0.6\textwidth,natwidth=80,natheight=60]{MosquitoLifeCycle.jpg}\label{fig:f1}
  \caption{Mosquito life cycle.[10]}
%  \label{Dengue vectors }
\end{figure} 
1. \textbf{Egg:} After taking a blood meal, female Aedes aegypti mosquitos produce on average 100 to 200 eggs per batch. The females can produce up to five batches of eggs during a lifetime. \\ %(The number of eggs is dependent on the size of the bloodmeal.)
Eggs are laid on damp surfaces in areas likely to temporarily flood, such as tree holes and man-made containers and a lot more places where rain-water collects or is stored.\\%(like barrels, drums, jars, pots, buckets, flower vases, plant saucers, tanks, discarded bottles, tins, tyres, water cooler, etc. ) \\
The female Aedes aegypti lays her eggs separately unlike most species. Not all eggs are laid at once, but they can be spread out over hours or days, depending on the availability of suitable substrates. Eggs will most often be placed at varying distances above the water line. The female mosquito will not lay the entire clutch at a single site, but rather spread out the eggs over several sites.\\
The eggs of Aedes aegypti are smooth, long, ovoid shaped, and roughly one millimeter long. When first laid, eggs appear white but within minutes turn a shiny black. In warm climates eggs may develop in as little as two days, whereas in cooler temperate climates, development can take up to a week. Laid eggs can survive for very long periods in a dry state, often for more than a year. However, they hatch immediately once submerged in water. This makes the control of the dengue virus mosquito very difficult.\\ 

2. \textbf{Larvae:} After hatching of the eggs, the larvae feed on organic particulate matter in the water, such as algae and other microscopic organisms. Most of the larval stage is spent at the water's surface. Larvae are often found around the home in puddles, tires, or within any object holding water.\\ %(although they will swim to the bottom of the container if disturbed or when feeding.) 
Larval development is temperature dependent. Males develop faster than females, so males generally pupate earlier. If temperatures are cool, Aedes aegypti can remain in the larval stage for months so long as the water supply is sufficient. \\
The larvae pass through four instars, spending a short amount of time in the first three, and up to three days in the fourth instar. Fourth instar larvae are approximately eight millimeters long. 
Males develop faster than females, so males generally pupate earlier. If temperatures are cool, Aedes aegypti can remain in the larval stage for months so long as the water supply is sufficient.\\ 
\begin{figure}[H]
  \centering
   \subfloat[Ae.aegypti larvae stage] {\includegraphics[width=0.3\textwidth,natwidth=50,natheight=50]{AA_larvae.jpg}\label{fig:f1}}
    \hfill
    \subfloat[Ae.aegypti pupae stage.]    {\includegraphics[width=0.24\textwidth,natwidth=60,natheight=50]{AA_pupa.jpg}\label{fig:f2}
}  
  \caption{Aedes aegypti larvae and pupae stage.}
  \label{Ae.Aegypti life stages.}    
\end{figure} 

3.\textbf{Pupae:} After the fourth instar, the larvae enters the pupal stage. Mosquito pupae are mobile and respond to stimuli. Pupae do not feed and take approximately two days to develop. Adults emerge by ingesting air to expand the abdomen thus splitting open the pupal case and emerge head first.[11]\\

4. Adult mosquitoes usually mate within a few days after emerging from the pupal stage. In most species, the males form large swarms, usually around dusk, and the females fly into the swarms to mate. Males typically live for about 5–7 days, feeding on nectar and other sources of sugar. After obtaining a full blood meal, the female will rest for a few days while the blood is digested and eggs are developed. This process depends on the temperature, but usually takes two to three days in tropical conditions. Once the eggs are fully developed, the female lays them and resumes host-seeking. \\

\textbf{Summary:} \\
\fbox{\begin{minipage}{42em}  
1. Aedes Aegypti goes through a complete metamorphosis with an egg, larvae, pupae, and adult stage.\\
2. After a blood meal,female Ae. aegypti mosquitoes lay upto five batches of 100 to 200 eggs. The eggs are laid on damped surface areas over several sites. The eggs can lie dormant in dry conditions for up to about nine months, after which they can hatch if exposed to     favourable conditions, i.e. water and food.[10] \\
3. Larval stage is temperature dependant. It passes through 4 instars. Males develop faster than females. If temperatures are cool, it remains in the larval stage for months as long as the water supply is sufficient.\\
4. Pupae do not feed and take approximately two days to develop into adult mosquito. The adult life span can range from two weeks to a month depending on environmental conditions.
\end{minipage}}\\ \\
%===================================================================

%=================================================================== 
\question
\label{7. Nucleation }
\textbf{What are the symptoms of dengue infection? What is the duration of mosquito bite to positive dengue condition? What are the intermediate stages of mosquito bite and positive dengue conditions?}\\
\textbf{Answer:} \\
1.\textbf{Incubation period:} (Time between mosquito’s bite to onset of symptoms)
After the infective bite, the dengue virus circulates in the blood. The incubation period for Dengue Fever and Dengue Haemorrhagic Fever lasts
from 3 to 14 days. The average incubation period is 4-6 days.[12] \\
\textbf{Symptoms of dengue infection:} The signs and symptoms of dengue fever vary according to the age of the patient.
The principal symptoms of dengue are high fever, severe headache, backache,
joint pains, nausea and vomiting, eye pain and rash.\\
Infants and young children may have a fever and a rash. They have milder illness
compared to other older children and adults. [12]\\
Thereafter symptoms appear such as a high fever (40°C/104°F) accompanied by 2 of the following symptoms: severe headache, pain behind the eyes, muscle and joint pains, nausea, vomiting, swollen glands or rash. These symptoms usually last for 2–7 days.[5]\\

2.\textbf{Dengue Haemorrhagic Fever (DHF) and Dengue Shock Syndrome (DSS):} These are
characterised by a sudden onset of fever, as high as 40-41ºC, lasting about
2-7 days with similar signs and symptoms of dengue fever. This is followed by
“leaky” capillaries allowing fluid component to escape from the blood vessels
causing shock and hemorrhagic manifestations such as bleeding nose or gums,
and possibly internal bleeding. Platelet counts will fall to less than 100,000/mm3.
The course of DHF is 7-10 days.\\
During the acute phase of dengue, it is difficult to distinguish DHF/DSS from
Dengue fever and other viral illnesses. The critical stage of DHF/DSS occurs most
frequently from 24 hours after the temperature falls to or below normal. If Dengue
Haemorrhagic Fever is not treated, it can lead to Dengue Shock Syndrome.\\

\textbf{Summary:} \\
\fbox{\begin{minipage}{42em}  
1. After the infective bite, the average incubation period of dengue virus is 3-6 days.\\
2. The principal symptoms of dengue are high fever, severe headache, backache,
joint pains, nausea and vomiting, eye pain and rash. \\
3. Severe dengue or Dengue Haemorrhagic Fever (DHF) is a potentially deadly complication due to plasma leaking, fluid accumulation, respiratory distress, severe bleeding, or organ impairment. \\
%4. If Dengue Haemorrhagic Fever is not treated, it can lead to Dengue Shock Syndrome.
\end{minipage}}\\ \\
%In patients with severe DHF or DSS, fever and non-specific signs and symptoms of a few days are followed by the sudden deterioration of the patient’s condition. The patient may appear lethargic at first and become restless.
%The signs of circulatory failure include:
%• low blood pressure
%• rapid and weak pulse
%• acute abdominal pain
%• cold clammy skin and restlessness
%Signs of haemorrhage are common and include bleeding under the skin (petechiae
%and purpura), gum bleeding, bleeding from the nose and gastrointestinal tract and
%heavy menstrual flow. They frequently experience acute abdominal pain shortly
%before the onset of shock.
%Severe dengue is a potentially deadly complication due to plasma leaking, fluid accumulation, respiratory distress, severe bleeding, or organ impairment. Warning signs occur 3–7 days after the first symptoms in conjunction with a decrease in temperature (below 38°C/100°F) and include: severe abdominal pain, persistent vomiting, rapid breathing, bleeding gums, fatigue, restlessness and blood in vomit. The next 24–48 hours of the critical stage can be lethal; proper medical care is needed to avoid complications and risk of death.

%===================================================================

%=================================================================== 
\question
\label{8. Dengue: Favourable habitats }
\textbf{ Which conditions are favourable for dengue mosquito breeding?}\\
\textbf{Answer:}\\
1. Aedes aegypti originated from Africa, is now present globally in tropical and sub-tropical regions (see also Epidemiology). The mosquito has a so-called cosmo-tropical distribution annually, and spreads to more temperate regions during the summer. \\
Living near man Aedes aegypti has become largely dependent on and adapted to humans. For instance, the mosquito has greatly reduced the `humming' sound it makes with their wings. Humans nearly hear Aedes aegypti, unlike other species whose humming is extremely irritating and awakens the deepest sleeper. The insect is very fast in flight unless gorged with blood. Other types of mosquito even fly into your face and can be easily caught or killed, not Aedes aegypti.\\

2. Aedes aegypti is adapted to breed around human dwellings and prefers to lay its eggs in clean water which contains no other living species. \\
The mosquito Aedes aegypti comes in three polytypic forms: domestic, sylvan, and peridomestic.
- The domestic form breeds in urban habitat, often around or inside houses.
- The sylvan form is a more rural form, and breeds in tree holes, generally in forests.
- The peridomestic form thrives in environmentally modified areas such as coconut groves and farms.\\

3. Mosquitoes prefer stagnant water within which to lay their eggs. They most commonly infest ponds, marshes, swamps and other wetland habitats. However, they are capable of thriving in a variety of locations and can successfully grow in numbers even when not in their natural habitat. Many species of mosquitoes use containers of water as egg-deposit sites.
Hot, humid environments are most amenable to mosquito growth and survival. Infestations can occur easily in tropical areas. \\ % Some species have also been known to inhabit freezing locations such as the Arctic Circle.
The aedes mosquito goes into hibernation when the temperature dips. The favourable condition for mosquito breeding is a humidity level of 60\% and an atmospheric temperature between 21 and 23 degrees celsius.[15]\\

Mosquito larvae can be found in various habitats. Some larvae are active in transient waters such as floodwater, ditches and woodland pools. The Anopheles, Culex, Culiseta, Coquillettidia and Uranotaenia species breed in permanent bodies of water and can survive in polluted water as well as freshwater, acid water and brackish water swamps. Other mosquito larvae may be present in container water sources such as puddles upon leaves and stagnant water within small pools.\\

\textbf{Relationship Between Mosquitoes and Water} \\
Mosquitoes live in water but must make periodic trips to the water’s surface in order to eliminate carbon dioxide and inhale a fresh supply of oxygen.\\
%The relationship between mosquitoes and water is different than non-aquatic types of insects. Generally, nature has two types of aquatic animals, those that live in water but get oxygen from the air (whales and mosquitoes) and those that live in water and get their oxygen from the water itself (fish). \\
Water provides mosquitoes with a place to lay eggs, grow and develop through their water stages (egg, larval and pupal). After the airborne portion of their lifecycle, females return to water to lay a new batch of fertile eggs. Female mosquitoes usually lay their eggs on the surface of water or in areas where water can rise, flood the eggs, and stimulate them to hatch. Even as adult mosquitoes leave the pupal stage and become adults, water still plays a role because adult mosquitoes exit the pupal case on the water’s surface and “dry out” before taking flight.[14] \\
%Mosquito breeding habits, the fertilization of the female by the male, normally occurs within a few days after the female has left the water source and is searching for, has taken or will soon take her first blood meal. While mosquito breeding may occur near a water source, mosquito breeding is one of the few activities that is not involved directly with a water source.Water is also a food source while mosquitoes are in their aquatic stages. Mosquitoes feed on the many kinds of particulate matter that occur in water. While water is the source of mosquito food, it also creates the possibility that aquatic-stage mosquitoes may become food for other animals, such as fish.[14]
4. National Environmental Agency, Singapore has listed following places at potential sites for mosquito breeding.[8]\\
  \textbf{Usual mosquito breeding sites:\\ }             
  1. Flower pot and flower pot plates. \\     
  2. Hardened soil of potted plants .\\       
  3. Corner of the toilet bowl. \\
  4. Gully trap. \\
  5. Roof gutter. \\
  6. Roadside drain. \\
  7. Scuppar drain. \\ 
    
  \textbf{Unsual mosquito breeding sites:\\} 
  1. Tree hole. \\
  2. Plant axil. \\
  3. Aircon-tray. \\
  4. BBQ pit. \\
  5. Canvas sheets. \\
  6. Discarded receptacles \\
  7. Planted box. \\

%\textbf{Summary:} \\
%\fbox{\begin{minipage}{42em}  
%1. After the infective bite, the average incubation period of dengue virus is 3-6 days.\\
%2. The principal symptoms of dengue are high fever, severe headache, backache,
%joint pains, nausea and vomiting, eye pain and rash. \\
%3. Severe dengue or Dengue Haemorrhagic Fever (DHF) is a potentially deadly complication due to plasma leaking, fluid accumulation, respiratory distress, severe bleeding, or organ impairment. \
%%4. If Dengue Haemorrhagic Fever is not treated, it can lead to Dengue Shock Syndrome.
%\end{minipage}}\\ \\
Table 1:Top 5 breeding habitats of Aedes Aegypti. \\

\begin{tabular}{ |p{7cm}|p{7cm}| }
\hline
\multicolumn{2}{|c|}{\textbf{Top 5 breeding habitats of Aedes Aegypti}} \\
\hline
Breeding habitats in homes & Breeding habitats in public places   \\
\hline
1. Domestic Containers & 1. Closed Perimeter Drains\\
2. Flower Pot Plates / Trays & 2. Discarded Receptacles\\
3. Ornamental Containers & 3. Gully Traps \\
4. Plants (Hardened Soil and Plant Axils) & 4. Opened Perimeter Drains\\
5. Toilet Bowl / Cistern & 5. HDB Corridor Scupper / Gullies \\
\hline
\end{tabular} \\ \\
%===================================================================

%=================================================================== 
%\question
%\label{9. Dengue: Peak time of breeding }
%\textbf{Which is the peak period and/or season of the year for dengue mosquito breeding and subsequent dengue hotspot/cluster emergence?}
%
%
%
%
%

%===================================================================

%=================================================================== 
\question
\label{10. Dengue: Prevention and Control }
\textbf{What are the preventive measures that can be implemented in order to avoid creating possible mosquito breeding site?} \\
\textbf{Answer:} \\
1. As per the WHO recommendation, the main method to control or prevent the transmission of dengue virus is to combat vector mosquitoes through:

Preventing mosquitoes from accessing egg-laying habitats by environmental management and modification. \\
Disposing of solid waste properly and removing artificial man-made habitats.\\
Covering, emptying and cleaning of domestic water storage containers on a weekly basis.\\
Applying appropriate insecticides to water storage outdoor containers.\\
Using of personal household protection such as window screens, long-sleeved clothes, insecticide treated materials, coils and vaporizers.\\
Improving community participation and mobilization for sustained vector control.\\
Applying insecticides as space spraying during outbreaks as one of the emergency vector-control measures.\\
Active monitoring and surveillance of vectors should be carried out to determine effectiveness of control interventions.\\
%It is crucial to prevent breeding of Ae. aegipty mosquitoes.\\

%===================================================================

%=================================================================== 
%\question
%\label{11. Dengue: Prognosis, Prevention and Control }
%\textbf{Is it possible to locate possible dengue clusters well before in time (mosquito breeding) and eradicate those locations? What information one needs to know to locate the possible dengue cluster?}\\
% Ref 1. Dengue Community Alert System
%NEA launched the Dengue Community Alert System in to display colour-coded banners (yellow, red or green banners) in dengue cluster areas to provide timely information to residents of the dengue situation in the precinct.

%===================================================================

%=================================================================== 
\question
\label{12. Dengue: Incapaciating Ae.Agypti }
\textbf{Is it possible to make dengue mosquitoes harmless at any stage of their life cycle before they spread virus? If yes, what are the possible ways?}\\
\textbf{Answer:}  \textbf{Wolbachia-Aedes Mosquito Suppression Strategy.}\\

NEA’s Environmental Health Institute (EHI) has studied various novel mosquito control methods over the past six years, and has focused on the Wolbachia method over the last four years. Wolbachia’s potential to reduce the Aedes aegypti mosquito population has been demonstrated in our laboratories.[27]\\ 

Wolbachia technology is considered a biological control method. Wolbachia are naturally occurring bacteria found in more than 60 per cent of insects around us, including butterflies, dragonflies, fruit flies, and various mosquito species such as Aedes albopictus, but not in the primary dengue vector mosquito Aedes aegypti. Wolbachia have not been shown to infect humans or other mammals, even when carried by biting insects.\\

The World Health Organization (WHO) Vector Control Advisory Group (VCAG) recommends carefully planned pilot deployment under operational conditions accompanied by rigorous independent monitoring and evaluation that builds entomological capacity to support operational use, for the use of Wolbachia against Aedes-borne diseases.\\

The Dengue Expert Advisory Panel (DEAP), comprising experts from Singapore, Australia, the United Kingdom (UK) and the United States of America (USA), recommends that Singapore explores the use of Wolbachia-carrying Aedes males to help suppress the Aedes mosquito population in Singapore, for further reduction of the risk of dengue.[26]\\

\textbf{How does Wolbachia-Aedes suppression strategy works?}\\
1. When male Wolbachia-carrying Aedes aegypti mosquitoes mate with female Aedes aegypti without Wolbachia, their resulting eggs do not hatch. This is because such matings are biologically incompatible. Thus release of male Wolbachia-Aedes aegypti will lead to a decline in the Aedes aegypti population in the field over time.\\
2. The outcome of this approach is consistent with our current emphasis on source reduction (removal of breeding habitats).\\
4. This mosquito suppression strategy is species-specific. Release of male Wolbachia-Aedes aegypti will only impact the Aedes aegypti population in the field, and not other insects.
[28]\\
%===================================================================

%=================================================================== 
%\question
%\label{13. Dengue: non-habitatas }
%\textbf{Which conditions are unfavourable for dengue mosquito breeding? \\ 
%Is it possible to convert a favourable breeding spot to unfavourable spot to stop the possible mosquito breeding?} \\
%
%%===================================================================
%
%%=================================================================== 
%\question
%\label{14. Dengue: Extinguishing mosquitoes}
%\textbf{What if all the mosquitoes in the world are extinguished? Will there be a missing link in the food chain? What would be the possible implications of this situation?} \\

%===================================================================
\textbf{Dengue numbers in Singapore.} \\ \\
The incidence of dengue in Singapore exhibits a distinct seasonality. It typically rises in the warmer months. Usually starting from April, the incidence reaches a peak in July or August, and declines in September or October.[16] 
WHO report mention that in Asia, Singapore has reported an increase in cases after a lapse of several years.[5]\\ \\ 
%=================================================================== 
\question
\label{15. Dengue: Hotspots in Singapore}
\textbf{Which are the dengue clusters in Singapore?} \\
\textbf{Answers:}\\
1. Operationally, a dengue cluster indicates a locality with active transmission where intervention is targeted.  It is formed when two or more cases have onset within 14 days and are located within 150m of each other (based on residential and workplace addresses as well as movement history). The clusters are categorised according to their current status. There are 3 alert levels:
\begin{figure}[H]
  \centering
   \includegraphics[width=0.8\textwidth,natwidth=510,natheight=542]{AlertLevels.png}
  \caption{Categories of dengue clusters.[17]}
   \label{fig:Categories of dengue clusters. }
\end{figure} 

2. Following map shows the high-risk dengue clusters in Singapore.
\begin{figure}[H]
  \centering
   \includegraphics[width=0.8\textwidth,natwidth=510,natheight=542]{HighRiskdengueclusters_SG.jpg}
  \caption{High-risk dengue clusters.[18]}
   \label{fig:High-risk dengue clusters. }
\end{figure} 

3. For 2016, about 17,000 mosquito breeding have been detected and destroyed. The breakdown of
breeding found at various premises types are found in Table 2. 
The largest proportion of breeding is still being found in homes.\\
Table 2: No. of breeding detected from Jan to Dec 2016* \\
\begin{tabular}{ |p{5cm}|p{5cm}| }
\hline
%\multicolumn{2}{|c|}{\textbf{No.of breeding detected from Jan to Dec 2016*}} \\
%\hline
Mosquito habitats & 2016  \\
\hline
Residential premises & About 7,800 (46\%)\\
Public areas About & 4,900 (30\%)\\
Construction sites & About 870 (5\%)\\
Others About & 3,200 (19\%)\\
Total About & 17,000\\
\hline
\end{tabular} \\ 

*Provisional; generated as of 9 Jan 2017.
%===================================================================

%=================================================================== 
%\question
%\label{16. Dengue: Hygiene and non-hygiene conditions }
%\textbf{What is the difference in hygiene conditions in dengue hotspot areas and non hotspot 
%    areas?}\\
%    
%===================================================================

%=================================================================== 
\question
\label{17. Dengue: Number of positive dengue cases}
\textbf{How many positive dengue cases are recorded in each district of Singapore?} \\
\textbf{Answer:}\\
In Singapore, there are around 4,000 to 5,000 reported cases of dengue or dengue haemorrhagic fever every year. Since 1980s, more than 50\% of deaths in the country have occurred in adults (individuals older than 15 years).[25]\\
MOH-NEA's quarterly dengue surveillance data mentions following information:[19]\\
1. \textbf{No. of cases reported}: 1. Jul to Sept 2016 (03/07/2016 - 1/10/2016), = 2,888 dengue cases (4 DHF cases) an increase of 2.7\% as compared to the previous quarter from April to Jun 2016. \\
                          2. Apr – Jun 2016 (03/04/2016-02/07/2016). \\
In 2016 (03/01/2016-31/12/2016), a cumulative total of \textbf{13,115} dengue
cases were notified to Ministry of Health (MOH).\\ 

2.\textbf{No. of deaths:} 1.Jan to Jun 2016 : 6 \\
                          2.Jul to Sept 2016 : 2 \\
                          3.Oct to Dec 2016 : 1\\
This brings the annual number of deaths in the year 2016 to 9. \\
                          
3. \textbf{Serotype Distribution:} Preliminary results of the positive dengue samples serotyped in Jul to Dec 2016 have indicated that DEN-1 accounted for majority of the typed samples, followed by DEN-2, DEN-3 and DEN-4. \\ 
\begin{figure}[H]
  \centering
   \includegraphics[width=0.8\textwidth,natwidth=510,natheight=542]{DengueCasesSG.png}
  \caption{Dengue Cases(Year 2013-2016)[21]}
   \label{Dengue Cases in Singapore 2013-2017}
\end{figure} 

\begin{figure}[H]
  \centering
   \includegraphics[width=0.8\textwidth,natwidth=510,natheight=542]{Jan2017_DengueCasesSG.png}
  \caption{Dengue Cases reported by E-week[21]} %(An epidemiological week (E-week) starts on a Sunday and ends on a Saturday)
   \label{Dengue Cases reported by E-week}
\end{figure} 
%===================================================================

%=================================================================== 
%\question
%\label{18. Dengue: Mortality record due to dengue }
%\textbf{How many deaths are recorded due to dengue in each year? What is the death rate due to dengue?} \\
%
%%===================================================================
%
%%=================================================================== 
%\question
%\label{19. Dengue: Treatment and total cure timeline }
%\textbf{What is the duration of positive dengue condition and complete recovery from dengue?\\ What are further precautions to be taken in order prevent getting infected with dengue again?} \\ 
%
%%===================================================================
%
%%=================================================================== 
%\question
%\label{20. Dengue: Types of treatments }
%\textbf{What treatments are used to treat a dengue patient? What is the cost of treatment and medication?}\\





%===================================================================

\end{questions}
%=================================================================== 

\begin{thebibliography}{28}
%Most lethal animal
\bibitem{Deadliest animal} 
KSL.com: 10 animals that cause the most human deaths. \\
gatesnotes: The blog of Bill Gates. 
\\\texttt{\url{https://www.ksl.com/?sid=29721017&nid=711}} \\

\bibitem{WHO} 
World Health Organization: A global brief on vector-borne diseases
\\\texttt{\url{http://apps.who.int/iris/bitstream/10665/111008/1/WHO_DCO_WHD_2014.1_eng.pdf}}\\
 
\bibitem{Healthline} 
Healthline: Mosquito, the most dangerous animal on Earth,
\\\texttt{\url{http://www.healthline.com/health-news/mosquitos-the-most-dangerous-animal-on-earth-021216\# 4}} \\
 
\bibitem{Wikipedia}  
Wikipedia: Mosquito-borne diseases,
\\\texttt{\url{https://en.wikipedia.org/wiki/Mosquito-borne_disease}} \\
 
%Lethal dengue
\bibitem{WHO} 
World Health Organization: Factsheets,
\\\texttt{\url{http://www.who.int/mediacentre/factsheets/fs117/en/}}\\
 
\bibitem{WHO} 
World Health Organization: World health report,
\\\texttt{\url{http://www.who.int/whr/1996/media_centre/executive_summary1/en/index9.html}} \\
 
%Dengue cause and transmission
\bibitem{Wikipedia} 
Wikipedia: Mosquito feeding,
\\\texttt{\url{https://en.wikipedia.org/wiki/Mosquito\# Feeding_by_adults}}\\
 
\bibitem{SG dengue website} 
National Environmental Agency, Singapore: Dengue website, 
\\\texttt{\url{http://www.dengue.gov.sg/subject.asp?id=12}} \\

%Interesting facts about mosquito feed and hunt for blood host.
\bibitem{Mosquito reviews} 
Elizabeth Miller: Mosquito diet, 
\\\texttt{\url{http://www.mosquitoreviews.com/mosquitoes-eat.html}}\\
 
%Lifecycle of Ae. Aegypti
\bibitem{SG dengue website} 
National Environmental Agency, Singapore: Dengue website, 
\\\texttt{\url{http://www.dengue.gov.sg/subject.asp?id=12}}\\

\bibitem{Denguevirusnet website} 
Dengue Virus net website: Life cycle of Aedes aegypti,
\\\texttt{\url{http://www.denguevirusnet.com/life-cycle-of-aedes-aegypti.html}}\\
 
\bibitem{Changhi general hospital} 
Changhi general hospital: All about dengue,
\\\texttt{\url{http://www.cgh.com.sg/Lists/Health\%20Library/Attachments/79/CGH\%20AllAbout\%20Dengue.pdf}
} \\
 
\bibitem{SG dengue website} 
National Environmental Agency, Singapore: Dengue website, 
\\\texttt{\url{http://www.dengue.gov.sg/subject.asp?id=100}}\\
 
\bibitem{Orkin} 
Orkin: Mosquito habitats,
\\\texttt{\url{http://www.orkin.com/other/mosquitoes/mosquito-habitats/}}\\
 
\bibitem{Times of India} 
TOI news: Dengue mosquito hibernation of dengue mosquitoes,
\\\texttt{\url{http://timesofindia.indiatimes.com/city/pune/Dengue-unrelenting-despite-fall-in-temp/articleshow/17328598.cms}}\\
 
\bibitem{ENB quarterly} 
Grace Yap, Shao-Hong Liang, Chanditha Hapuarachchi, Chee-Seng Chong, Li-Kiang Tan, Xu Liu, Yuan Shi, Kangwei Zeng, Carmen Koo, Alex Cook, Yee-Ling Lai, Lee-Ching Ng.
\textit{Dengue outlook for Singapore in 2016.}
Epidemiological news bulletin (ENB quarterly), Singapore: Vol.42, 1 Jan 2016.

\bibitem{SG dengue website} 
National Environmental Agency, Singapore: Dengue website,
\\\texttt{\url{http://www.dengue.gov.sg/subject.asp?id=74}}\\
 
\bibitem{Straitstimes}
Straitstimes: Dengue and zika hotspots in Singapore, 
\\\texttt{\url{http://www.straitstimes.com/singapore/sims-place-is-hot-spot-for-zika-and-dengue}}\\
 
\bibitem{SG dengue website} 
National Environmental Agency, Singapore.
\textit{MOH-NEA quarterly dengue surveillance data, Jul-Sept 2016}.
17th Oct 2016.

 
\bibitem{SG dengue website} 
National Environmental Agency, Singapore.
\textit{MOH-NEA quarterly dengue surveillance data, Oct-Dec 2016}.
16th Jan 2017.
%\\\texttt{\url{http://www.dengue.gov.sg/pdf/Q4\%20Dengue\%20Surveillance\%20Data.pdf}}\\
 
\bibitem{SG dengue website} 
National Environmental Agency, Singapore: Dengue website,
\\\texttt{\url{http://www.dengue.gov.sg/subject.asp?id=73}}\\

\bibitem{NCBI}
Anne T B, Ake Lundkvist.
\textit{Dengue viruses: An overview}.
Infect Ecol Epidemiol. 2013; 3: 10.3402/iee.v3i0.19839.
%\\\texttt{\url{https://www.ncbi.nlm.nih.gov/pmc/articles/PMC3759171/}}\\
 
\bibitem{Nature journal} 
Nature: Dengue viruses,
\\\texttt{\url{http://www.nature.com/scitable/topicpage/dengue-viruses-22400925}}\\
 
\bibitem{Wikipedia} 
Wikipedia: Dengue virus,
\\\texttt{\url{https://en.wikipedia.org/wiki/Dengue_virus}}\\
 
\bibitem{Distribution of dengue fever} 
Ridpest.com: Geographical distribution of dengue fever,
\\\texttt{\url{https://caring.ridpest.com/the-geographical-distribution-of-dengue-fever/}}\\
 
\bibitem{SG Dengue website} 
National Environmental Agency, Singapore: Dengue website,
\\\texttt{\url{http://www.dengue.gov.sg/subject.asp?id=163}} \\
 
\bibitem{NEA} 
National Environmental Agency, Singapore: Environmental Public Health Research, 
\\\texttt{\url{http://www.nea.gov.sg/public-health/environmental-public-health-research/wolbachia-technology}} \\
 
\bibitem{NEA}
National Environmental Agency, Singapore: Wolbachia Technology,  
\\\texttt{\url{http://www.nea.gov.sg/public-health/environmental-public-health-research/wolbachia-technology/wolbachia-aedes-mosquito-suppression-strategy-how-it-works}} \\

\end{thebibliography}
\end{document}  